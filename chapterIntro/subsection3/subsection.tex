\subsection{Основы потока Риччи}

Гамильтон \cite{c29} ввёл уравнение потока Риччи.
\[
\frac{\partial g(t)}{\partial t} = -2Ric(g(t))
\]
Это уравнение эволюции для однопараметрического семейства 
римановых метрик $g(t)$ на гладком многообразии $M$. 
Уравнение потока Риччи является слабо параболическим и 
строго параболическим с учётом «группы калибровок», 
которая представляет собой группу диффеоморфизмов 
исходного гладкого многообразия. 
Это уравнение следует рассматривать как нелинейную, 
тензорную версию уравнения теплопроводности. 
Из него можно вывести уравнение эволюции для римановой метрики, 
тензора Риччи и функции скалярной кривизны. 
Все эти уравнения являются параболическими. 
Например, уравнение эволюции для скалярной кривизны 
$R(x,t)$ выглядит следующим образом:
\begin{equation}
    \frac{\partial R}{\partial t}(x,t) = \bigtriangleup R(x,t)+2{| Ric(x,t)|}^2
    \label{eqn0.1}
\end{equation}
иллюстрируя сходство с уравнением теплопроводности. 
(Здесь $\bigtriangleup$ — это лапласиан с неположительным спектром.)\\

\textbf{3.1. Первые результаты.} Конечно, первые результаты, которые нам нужны, 
— это уникальность и существование решений уравнения потока Риччи для 
компактных многообразий в короткий промежуток времени. Эти результаты 
были доказаны Гамильтоном (\cite{c29}) с использованием теоремы обратной функции 
Нэша-Мозера (\cite{c28}). Эти результаты являются стандартными для строго 
параболических уравнений. На данный момент существует достаточно стандартный 
метод работы «модуло» группы калибровок (группы диффеоморфизмов), что 
позволяет достичь строго параболической ситуации, в которой применяются 
классические результаты существования, уникальности и гладкости. Метод для 
уравнения потока Риччи называется «трюк де Тёрка».

Существует также результат, который позволяет сшить локальные решения 
$(U,g(t)),a\leq t\leq b$ и $(U,h(t)),b\leq t\leq c$ в одно гладкое решение, 
определённое на интервале $a\leq t\leq c$, при условии, что 
$g(b)=h(b)$.

Заданное уравнение потока Риччи $(M,g(t))$ можно всегда сдвинуть, 
заменив $t$ на $t+t_0$ для некоторого фиксированного $t_0$, что даст 
новый поток Риччи. Также можно масштабировать его на любое 
положительное число $Q$, задав $h(t)=Q g(Q^{-1} t)$, 
что создаст новый поток Риччи.