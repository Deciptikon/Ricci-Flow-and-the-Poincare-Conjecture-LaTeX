\subsection{Основы потока Риччи}

Гамильтон \cite{c29} ввёл уравнение потока Риччи.
\[
\frac{\partial g(t)}{\partial t} = -2Ric(g(t))
\]
Это уравнение эволюции для однопараметрического семейства 
римановых метрик $g(t)$ на гладком многообразии $M$. 
Уравнение потока Риччи является слабо параболическим и 
строго параболическим с учётом «группы калибровок», 
которая представляет собой группу диффеоморфизмов 
исходного гладкого многообразия. 
Это уравнение следует рассматривать как нелинейную, 
тензорную версию уравнения теплопроводности. 
Из него можно вывести уравнение эволюции для римановой метрики, 
тензора Риччи и функции скалярной кривизны. 
Все эти уравнения являются параболическими. 
Например, уравнение эволюции для скалярной кривизны 
$R(x,t)$ выглядит следующим образом:
\begin{equation}
    \frac{\partial R}{\partial t}(x,t) = \bigtriangleup R(x,t)+2{| Ric(x,t)|}^2
    \label{eqn0.1}
\end{equation}
иллюстрируя сходство с уравнением теплопроводности. 
(Здесь $\bigtriangleup$ — это лапласиан с неположительным спектром.)