\subsection{Основы потока Риччи}

Гамильтон \cite{c29} ввёл уравнение потока Риччи.
\[
\frac{\partial g(t)}{\partial t} = -2Ric(g(t))
\]
Это уравнение эволюции для однопараметрического семейства 
римановых метрик $g(t)$ на гладком многообразии $M$. 
Уравнение потока Риччи является слабо параболическим и 
строго параболическим с учётом «группы калибровок», 
которая представляет собой группу диффеоморфизмов 
исходного гладкого многообразия. 
Это уравнение следует рассматривать как нелинейную, 
тензорную версию уравнения теплопроводности. 
Из него можно вывести уравнение эволюции для римановой метрики, 
тензора Риччи и функции скалярной кривизны. 
Все эти уравнения являются параболическими. 
Например, уравнение эволюции для скалярной кривизны 
$R(x,t)$ выглядит следующим образом:
\begin{equation}
    \frac{\partial R}{\partial t}(x,t) = \bigtriangleup R(x,t)+2{| Ric(x,t)|}^2
    \label{eqn0.1}
\end{equation}
иллюстрируя сходство с уравнением теплопроводности. 
(Здесь $\bigtriangleup$ — это лапласиан с неположительным спектром.)\\

\textbf{3.1. Первые результаты.} Конечно, первые результаты, которые нам нужны, 
— это уникальность и существование решений уравнения потока Риччи для 
компактных многообразий в короткий промежуток времени. Эти результаты 
были доказаны Гамильтоном (\cite{c29}) с использованием теоремы обратной функции 
Нэша-Мозера (\cite{c28}). Эти результаты являются стандартными для строго 
параболических уравнений. На данный момент существует достаточно стандартный 
метод работы «модуло» группы калибровок (группы диффеоморфизмов), что 
позволяет достичь строго параболической ситуации, в которой применяются 
классические результаты существования, уникальности и гладкости. Метод для 
уравнения потока Риччи называется «трюк де Тёрка».

Существует также результат, который позволяет сшить локальные решения 
$(U,g(t)),a\leq t\leq b$ и $(U,h(t)),b\leq t\leq c$ в одно гладкое решение, 
определённое на интервале $a\leq t\leq c$, при условии, что 
$g(b)=h(b)$.

Заданное уравнение потока Риччи $(M,g(t))$ можно всегда сдвинуть, 
заменив $t$ на $t+t_0$ для некоторого фиксированного $t_0$, что даст 
новый поток Риччи. Также можно масштабировать его на любое 
положительное число $Q$, задав $h(t)=Q g(Q^{-1} t)$, 
что создаст новый поток Риччи.\\

\textbf{3.2. Градиентные сжимающиеся солитоны.} Предположим, что 
$(M,g)$ — полное риманово многообразие, и существует константа 
$\lambda >0$, такая что:
\[
    Ric(g)=\lambda g
\]
В этом случае легко увидеть, что существует поток Риччи, задаваемый:
\[
    g(t)=(1-2\lambda t)g
\]
В частности, все метрики в этом потоке отличаются на постоянный множитель, 
зависящий от времени, и метрика является убывающей функцией времени. 
Такие решения называются \textit{сжимающимися солитонами}. Примеры включают 
компактные многообразия с постоянной положительной кривизной Риччи.

Существует тесно связанный с ними, но более общий класс примеров: 
\textit{градиентные сжимающиеся солитоны}. Предположим, что 
$(M,g)$ — полное риманово многообразие, и существует константа 
$\lambda >0$ и функция $f:M \rightarrow \mathbb{R}$, удовлетворяющие следующему условию:
\[
    Ric(g)=\lambda g - {Hess}^g f
\]
В данном случае существует поток Риччи, представляющий собой 
семейство сжимающихся метрик после применения обратного 
преобразования посредством одномерного семейства диффеоморфизмов, 
порожденного зависящим от времени векторным полем 
$\frac{1}{1-2\lambda t}{\bigtriangledown}_g f$. 
Примером градиентного убывающего солитона является многообразие 
$S^2 \times \mathbb{R}$ с семейством метрик, представляющим 
собой произведение семейства сжимающихся сферических метрик на 
$S^2$ и семейства постоянных стандартных метрик на $\mathbb{R}$. 
Функция $f$ в этом случае равна $s^2/4$, 
где $s$ — евклидова координата на $\mathbb{R}$.\\

\textbf{3.3. Управление высшими производными кривизны.} Теперь 
давайте обсудим результаты гладкости для геометрических пределов. 
Общий результат в этом направлении — теорема Ши, см.\cite{c65, c66}. 
Это стандартный тип результата для параболических уравнений. 
Конечно, ситуация здесь осложняется существованием группы калибровок. 
В общем, теорема Ши говорит следующее. Пусть  
$B(x,t_0,r)$ — метрический шар в $(M,g(t_0))$, с центром в точке  
$x$ и радиусом $r$. Если мы можем контролировать норму тензора 
кривизны в обратной окрестности формы $B(x,t_0,r)\times [0,t_0]$, 
тогда для любого $k>0$ мы можем контролировать $k$-ую 
ковариантную производную кривизны в 
$B(x,t_0,r/2^k)\times [0,t_0]$ как константу превышающую 
$t^{k/2}$. Этот результат имеет много важных последствий для 
нашего исследования, так как он говорит нам, что геометрические 
пределы — это пределы обладающие гладкостью. 
Возможно, первое важное утверждение, которое 
стоит выделить, — это факт (установленный ранее Хэмилтоном), 
что если $(M,g(t))$ — это поток Риччи, 
определенный на $0 \leq t< T< \infty$, и если тензор кривизны Риманна 
ограничен на протяжении всего потока, тогда поток Риччи 
продолжается после времени $T$. 

В третьей главе этот материал пересматривается, и при 
необходимости представлены небольшие варианты результатов и 
аргументов из литературы.\\

\textbf{3.4. Обобщение потоков Риччи.} Поскольку мы не 
можем ограничиться только потоками Риччи, и должны 
рассматривать более общие объекты, такие как потоки Риччи 
с хирургией, важно установить основные аналитические 
результаты и оценки в контексте, более общем, чем потоки 
Риччи. Мы решили сделать это в контексте обобщенных 
потоков Риччи.

Обобщенный трехмерный поток Риччи состоит из гладкого 
четырехмерного многообразия $\mathcal{M}$ (пространства-времени) 
с функцией времени $\textbf{t}:\mathcal{M} \rightarrow \mathbb{R}$ 
и гладким векторным полем $\chi$. 
Такими, что должны выполнятся следующие условия:
\begin{enumerate}
    \item Каждое $x\in \mathcal{M}$ имеет окрестность вида $U \times J$, где 
    $U$ — открытое подмножество в $\mathbb{R}^3$, а $J\subset \mathbb{R}$ — интервал, 
    в котором $\textbf{t}$ является проекцией на $J$, а $\chi$ — единичное 
    векторное поле, касательное к одномерному слоению 
    $\{ u\} \times J$, направленному в сторону возрастания $\textbf{t}$.
    Мы называем $\textbf{t}^{-1}(t)$ $t$-временным срезом. Это гладкое 3-многообразие.
    \item Образ $\textbf{t}(\mathcal{M})$ — это связный интервал $I$ в $\mathbb{R}$, возможно бесконечный. 
    Граница $\mathcal{M}$ — это прообраз границы $I$ при обратном отображении $\textbf{t}$.
    \item Множество уровней  $\textbf{t}^{-1}(t)$  образуют слоение коразмерности один 
    на $\mathcal{M}$, называемое горизонтальным слоением, где границы компонентов 
    $\mathcal{M}$ являются листьями.
    \item Существует метрика $G$ на горизонтальном распределении, 
    т.е. на распределении касательном к множеству уровней $\textbf{t}$.
    Эта метрика задаёт риманову метрику на каждом $t$ временном срезе, 
    которая изменяется гладко по мере изменения самого временного среза.
    Мы определяем кривизну $G$ в точке $x \in \mathcal{M}$ как кривизну 
    римановой метрики, индуцированной $G$ на временном срезе $\mathcal{M}_t$ в $x$.
    \item Благодаря первому свойству интегральные кривые векторного 
    поля $ \chi $ сохраняют горизонтальное слоение, а значит, и горизонтальное 
    распределение. Таким образом, мы можем взять производную Ли от $ G $ 
    вдоль $ \chi $. Уравнение потока Риччи тогда принимает вид:
    \[
    \mathcal{L}_{\chi}(G)=-2Ric(G)
    \]
\end{enumerate}

Локально в пространстве-времени горизонтальная метрика 
представляет собой простое гладко меняющееся семейство 
римановых метрик на многообразии фиксированной гладкости, 
а уравнение эволюции является обычным уравнением потока 
Риччи. Это означает, что стандартные формулы для эволюции 
кривизн, а также большая часть аналитического анализа 
потоков Риччи сохраняются в этом обобщённом контексте. 
В конечном счёте поток Риччи с хирургией является более 
сингулярным типом пространства-времени, но он будет иметь 
открытое плотное подмножество, которое является 
обобщённым потоком Риччи, и все аналитические оценки 
выполняются в этом открытом подмножестве.

Понятие канонических окрестностей имеет смысл в контексте 
обобщённых потоков Риччи. Существует также понятие сильной 
$ \epsilon $-шейки. Рассмотрим вложение 
$ \psi : S^2 \times (-\epsilon^{-1}, \epsilon^{-1}) \times (-1, 0] $ 
в пространство-время такое, что функция времени испытывает pullback 
на проекцию в $ (-1, 0] $, а векторное поле $ \chi $ 
pullback на $ \partial / \partial t $. Если существует 
такое вложение в соответствующим образом смещённую и 
перемасштабированную версию исходного обобщённого потока Риччи, 
что pullback перемасштабированной горизонтальная метрики 
находится в пределах $ \epsilon $ в $ C^{[1/\epsilon]} $-топологии 
произведения сжимающегося семейства окружностей $ S^2 $ с евклидовой 
метрикой на $ (-\epsilon^{-1}, \epsilon^{-1}) $, то мы говорим, 
что $ \psi $ является сильной $ \epsilon $-шейкой в обобщённом 
потоке Риччи.\\

\textbf{3.5. Принцип максимума.} Уравнение потока Риччи ...
