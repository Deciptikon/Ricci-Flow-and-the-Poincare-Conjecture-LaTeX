\subsection{Основы геометрии Римана}

\textbf{2.1. Объем и радиус инъективности.} Одним важным общим понятием, которое 
используется в дальнейшем, является понятие того, что многообразие не схлопывается 
в какой-то точке. Пусть у нас есть точка $x$ в полном римановом $n$-мерном многообразии. 
Мы говорим, что многообразие \textit{$\kappa $-не-коллапсируемо} в точке $x$, если 
выполняется следующее условие: для любого $r$, при котором норма тензора кривизны Римана
$| Rm| $ не превышает $r^{-2}$ во всех точках метрического шара $B(x,r)$ радиуса $r$, центрированного 
в $x$, выполняется Vol$B(x,r)\geq \kappa r^{n}$. Существует связь между этим понятием и радиусом 
инъективности многообразия $M$ в точке $x$. А именно, если $| Rm| \leq r^{-2}$ на $B(x,r)$ и если 
$B(x,r)$ \textit{$\kappa $-не-коллапсируемо}, то радиус инъективности многообразия $M$ в точке 
$x$ больше или равен положительной константе, которая зависит только от $r$ и $\kappa $. 
Преимущество работы с условием объема, не приводящим к схлопыванию, заключается в том, 
что, в отличие от радиуса инъективности, существует простое уравнение для эволюции объема 
под действием потока Риччи.

Еще одним важным общим результатом является теорема о сравнении объемов Бишопа-Громова, 
которая утверждает, что если кривизна Риччи полного риманова $n$-мерного многообразия $M$ 
ограничена снизу постоянной $(n-1)K$, то для любой точки $x\in M$ отношение объема шара 
$B(x,r)$ к объему шара радиуса $r$ в пространстве постоянной кривизны $K$ является 
невозрастающей функцией, предел которой при $r\rightarrow 0$ равен $1$. 

Все эти базовые факты из римановой геометрии рассматриваются в первой главе.\\


\textbf{2.2. Многообразия с неотрицательной кривизной.} По причинам, которые должны стать 
ясными из вышеизложенного и, в любом случае, станут еще более очевидными вскоре, 
многообразия с неотрицательной кривизной играют чрезвычайно важную роль в анализе 
потоков Риччи с хирургией. Нам нужно несколько общих результатов о них. 
Первый — это теорема о душе для многообразий с неотрицательной секционной кривизной. 
\textit{Душа} — это компактное, тотально геодезическое подмногообразие. Все многообразие 
диффеоморфно полному пространству векторного расслоения над любой из своих душ. 
Если некомпактное $n$-мерное многообразие имеет положительную секционную кривизну, 
то любая душа для него — это точка, и, в частности, это многообразие диффеоморфно 
евклидову пространству. Кроме того, функция расстояния $f$ от души имеет свойство, 
что для каждого $t > 0$ прообраз $f^{-1} (t)$ гомеоморфен ($n-1$)-сфере, а прообраз под 
этой функцией расстояния любого недегенерированного интервала $I\subset R^{+}$ гомеоморфен 
$S^{n-1} \times I$.