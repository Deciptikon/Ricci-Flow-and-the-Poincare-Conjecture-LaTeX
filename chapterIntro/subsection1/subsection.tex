\subsection{Обзор аргументов Перельмана}
В размерностях, меньших или равных трём, любая риманова метрика с 
постоянной кривизной Риччи имеет постоянную секционную кривизну. 
Классические результаты в римановой геометрии показывают, что 
универсальное покрытие замкнутого многообразия с постоянной 
положительной кривизной диффеоморфно сфере, а фундаментальная 
группа идентифицируется с конечной подгруппой ортогональной группы, 
которая действует линейно и свободно на универсальном покрытии. 
Таким образом, можно подойти к гипотезе Пуанкаре и более общей 
проблеме 3-мерных сферических пространственных форм, задав следующий 
вопрос. С учетом соответствующих предположений о фундаментальной 
группе 3-мерного многообразия $M$, как установить существование 
метрики с постоянной кривизной Риччи на $M$? 
Основным элементом для создания такой метрики является уравнение 
потока Риччи, введённое Ричардом Гамильтоном в [29]:
\[
\frac{\partial g(t)}{\partial t} = -2Ric(g(t))
\]
