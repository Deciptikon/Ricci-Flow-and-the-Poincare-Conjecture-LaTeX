\subsection{Обзор аргументов Перельмана}
В размерностях, меньших или равных трём, любая риманова метрика с 
постоянной кривизной Риччи имеет постоянную секционную кривизну. 
Классические результаты в римановой геометрии показывают, что 
универсальное покрытие замкнутого многообразия с постоянной 
положительной кривизной диффеоморфно сфере, а фундаментальная 
группа идентифицируется с конечной подгруппой ортогональной группы, 
которая действует линейно и свободно на универсальном покрытии. 
Таким образом, можно подойти к гипотезе Пуанкаре и более общей 
проблеме 3-мерных сферических пространственных форм, задав следующий 
вопрос. С учетом соответствующих предположений о фундаментальной 
группе 3-мерного многообразия $M$, как установить существование 
метрики с постоянной кривизной Риччи на $M$? 
Основным элементом для создания такой метрики является уравнение 
потока Риччи, введённое Ричардом Гамильтоном в [29]:
\[
\frac{\partial g(t)}{\partial t} = -2Ric(g(t))
\]
Где $Ric(g(t))$ — это кривизна Риччи метрики $g(t)$. Фиксированные точки 
(с точностью до масштабирования) этого уравнения — римановы метрики с 
постоянной кривизной Риччи. Для общего введения в тему потоков Риччи 
см. обзорную статью Гамильтона [34], книгу Чоу и Кнопфа [13] или 
книгу Чоу, Лу и Ни [14].
Уравнение потока Риччи — это (слабо) параболическое уравнение в 
частных производных для римановых метрик на гладком многообразии. 
Следуя Гамильтону, поток Риччи определяется как семейство римановых 
метрик $g(t)$ на фиксированном гладком многообразии, параметризованное параметром 
$t$ на некотором интервале, которое удовлетворяет этому уравнению. 
Параметр $t$ интерпретируется как время, а уравнение рассматривается 
как задача Коши: начиная с любого риманова многообразия $(M, g_{0})$, 
требуется найти поток Риччи с начальной метрикой $g_{0}$; то есть найти 
однопараметрическое семейство $(M, g(t))$ римановых многообразий, 
где $g(0) = g_{0}$, удовлетворяющее уравнению потока Риччи. 
Это уравнение справедливо в любых размерностях, однако здесь мы 
сосредотачиваемся на трёхмерном случае.
Вкратце, метод доказательства заключается в том, чтобы начать с 
любой римановой метрики на заданном гладком 3-мерном многообразии 
и применить к ней поток Риччи, чтобы получить метрику с постоянной 
кривизной, которую мы ищем. Есть два примера, где этот метод 
работает именно так, оба принадлежат Гамильтону.\\
(i) Если начальная метрика имеет положительную кривизну Риччи, 
Гамильтон доказал более 20 лет назад ([29]), что при потоке Риччи 
многообразие сжимается до точки за конечное время, то есть возникает 
сингулярность в конечное время, и, по мере приближения к сингулярности, 
диаметр многообразия стремится к нулю, а кривизна стремится к 
бесконечности в каждой точке. Гамильтон также показал, что в этом 
случае скейлинг метрики с помощью зависящей от времени функции, 
так чтобы диаметр оставался постоянным, приводит к однопараметрическому 
семейству метрик, которые гладко сходятся к метрике с постоянной 
положительной кривизной.\\
(ii) На другом крайнем случае, в работе [36] Гамильтон показал, что если 
поток Риччи существует для любого времени и если выполняются 
соответствующие ограничения на кривизну, а также другое геометрическое 
ограничение, то при $t \rightarrow \infty$, после скейлинга для фиксированного 
диаметра, метрика сходится к метрике с постоянной отрицательной кривизной.

Результаты в общем случае значительно сложнее для формулировки и намного 
труднее для доказательства. Хотя Гамильтон установил, что уравнение потока 
Риччи обладает свойствами краткосрочного существования, то есть можно определить 
$g(t)$ для $t$ на интервале $[0,T)$, где $T$ зависит от начальной метрики, оказывается, что 
если топология многообразия достаточно сложна, например, оно представляет собой 
нетривиальную связанную сумму, то независимо от начальной метрики неизбежно 
возникнут сингулярности за конечное время, вызванные топологией. Более того, 
даже если многообразие имеет простую топологию, начиная с произвольной метрики, 
можно ожидать (и нельзя исключить возможность) возникновения сингулярностей за 
конечное время в потоке Риччи. Эти сингулярности, в отличие от случая положительной 
кривизны Риччи, появляются на правильных подмножествах многообразия, 
а не на всем многообразии.
Таким образом, для вывода топологических последствий, указанных выше, в общем 
случае недостаточно остановить анализ в момент появления первой сингулярности 
в потоке Риччи. Это приводит к изучению более общего процесса эволюции, 
называемого \textit{потоком Риччи с хирургией}, впервые введенного Гамильтоном в контексте 
четырехмерных многообразий. Этот процесс эволюции также параметризован интервалом 
времени, так что для каждого $t$ из интервала определения существует компактное 
риманово трехмерное многообразие $M_{t}$. Однако существует дискретный набор моментов 
времени, в которых многообразия и метрики претерпевают топологические и 
метрические разрывы (хирургии). В каждом из интервалов между такими моментами 
эволюция подчиняется обычному потоку Риччи, но из-за хирургии топологический 
тип многообразия $M_{t}$ меняется при переходе из одного интервала в следующий.
С аналитической точки зрения хирургии в моменты разрывов вводятся для того, 
чтобы `вырезать' окрестности сингулярностей по мере их развития и вручную 
вставить вместо `вырезанных'  областей геометрически правильные регионы. 
Это позволяет продолжить (точнее, перезапустить) поток Риччи с новой метрикой, 
построенной в момент разрыва. Конечно, процесс хирургии также изменяет топологию. 
Чтобы можно было сделать полезные топологические выводы о таком процессе, 
необходимо как иметь результаты о потоке Риччи, так и контролировать топологию и 
геометрию процесса хирургии в моменты сингулярностей. Например, важно для 
топологических приложений, чтобы хирургии выполнялись вдоль 2-сфер, а не 
поверхностей более высокого рода. Хирургия вдоль 2-сфер создает разложение на 
связанные суммы, которое хорошо изучено в топологии, в то время как, например, 
операции Дена вдоль торов могут полностью разрушить топологию, превращая любое 
трехмерное многообразие в любое другое.

Изменение топологии оказывается полностью понятным, и, что удивительно, 
процессы хирургии создают именно те топологические операции, которые необходимы, 
чтобы разрезать многообразие на части, где поток Риччи может сформировать 
метрики, достаточно контролируемые для распознавания топологии.

Основная часть этой книги (главы 1–17 и приложение) посвящена установлению 
следующего результата о долгосрочном существовании потока Риччи с хирургией.\vspace{0.5em}

Теорема 0.3. \textit{Пусть $(M, g_{0})$ — замкнутое риманово трехмерное 
многообразие. Предположим, что в $M$ нет вложенных, локально разделенных областей 
$RP^{2}$. Тогда существует поток Риччи с хирургией, 
определенный для всех $t\in [0,\infty)$ с начальной метрикой $(M, g_{0})$. 
Множество моментов разрывов для этого потока Риччи с хирургией является 
дискретным подмножеством $[0,\infty)$.
Топологическое изменение трехмерного многообразия при прохождении через 
момент хирургии представляет собой разложение на связанную сумму с 
удалением связных компонент, каждая из которых диффеоморфна одному из 
следующих многообразий: $S^{2}\times S^{1}, RP^{3} \# RP^{3}$, не-ориентируемое 
расслоение над $S^{1}$ с базой $S^{2}$, или многообразие, допускающее 
метрику с постоянной положительной кривизной.}\vspace{0.5em}

Хотя Теорема 0.3 является ключевой для всех приложений потока Риччи к 
топологии трёхмерных многообразий, доказательство для трёхмерных многообразий, 
описанных в Теореме 0.1, упрощено и избегает любых ссылок на поведение потока 
при стремлении времени к бесконечности благодаря следующему результату о 
конечновременном вырождении.\vspace{0.5em}

Теорема 0.4. \textit{Пусть $M$ — замкнутое 3-многообразие, фундаментальная группа которого 
является свободным произведением конечных групп и бесконечных циклических групп. 
Пусть $g_{0}$ — произвольная риманова метрика на $M$. Тогда $M$ не содержит локально 
разделяющихся $RP^{2}$, так что поток Риччи с хирургией, определённый для любого 
положительного времени, может быть задан, начиная с метрики $(M,g_{0})$, как описано 
в Теореме 0.3. Этот поток Риччи с хирургией завершает своё существование через 
конечное время $T<\infty$ в том смысле, что многообразия $M_{t}$ пусты для всех $t \geq T$.}\vspace{0.5em}

Этот результат устанавливается в главе 18, следуя доводу, предложенному Перельманом 
в [54], см. также [15].

Мы немедленно выводим Теорему 0.1 из Теорем 0.3 и 0.4 следующим образом:
Пусть $M$ — 3-многообразие, удовлетворяющее гипотезе Теоремы 0.1. Тогда 
существует конечная последовательность $M=M_{0},M_{1},…,M_{k}=\emptyset $, такая, что для 
каждого $i$, $1\leq i\leq k$, $M_{i}$ получается из $M_{i-1}$ либо с помощью разложения на связную 
сумму, либо путём удаления компоненты, диффеоморфной одной из следующих: 
$S^{2}\times S^{1}$ , $RP^{3} \# RP^{3}$ , неориентируемого расслоения с базой $S^{1}$ и слоем 
$S^{2}$, или трёхмерной сферической пространственной формы.

Очевидно, по нисходящей индукции по $i$ следует, что каждая связная компонента 
$M_{i}$ диффеоморфна связной сумме трёхмерных сферических пространственных форм, 
копий $S^{2} \times S^{1}$ и копий неориентируемого расслоения с базой $S^{1}$ и слоем 
$S^{2}$. В частности, $M=M_{0}$ имеет эту форму. Так как $M$ предположительно связно, 
это доказывает теорему. Более того, этот довод доказывает следующее: