\documentclass{article}
\usepackage[utf8]{inputenc} % Кодировка UTF-8
\usepackage[T2A]{fontenc}   % Кодировка для кириллицы
\usepackage[russian]{babel} % Поддержка русского языка
\usepackage[colorlinks=true,linkcolor=blue]{hyperref}
\usepackage{tocloft} % Для настройки оглавления
\usepackage{footnotehyper}

% Убираем точки для всех уровней содержания
\renewcommand{\cftsecleader}{\cftdotfill{\cftnodots}} % Для разделов
\renewcommand{\cftsubsecleader}{\cftdotfill{\cftnodots}} % Для подразделов (если есть)

% Настраиваем нумерацию в содержании
\renewcommand{\thesubsection}{\arabic{subsection}}
\renewcommand{\thesection}{Глава \arabic{section}.}
\renewcommand{\cftsecaftersnum}{ }
\cftsetindents{section}{0em}{5.0em}

\begin{document}

\begin{center}
    \huge\textbf{Поток Риччи и гипотеза Пуанкаре}
    \vspace{1cm}

    \huge{Джон В. Морган и Ганг Тянь}
\end{center}
\newpage
\newpage
\tableofcontents
\newpage
\section*{Введение}
\addcontentsline{toc}{section}{Введение}
\subsection{Обзор аргументов Перельмана}
В размерностях, меньших или равных трём, любая риманова метрика с 
постоянной кривизной Риччи имеет постоянную секционную кривизну. 
Классические результаты в римановой геометрии показывают, что 
универсальное покрытие замкнутого многообразия с постоянной 
положительной кривизной диффеоморфно сфере, а фундаментальная 
группа идентифицируется с конечной подгруппой ортогональной группы, 
которая действует линейно и свободно на универсальном покрытии. 
Таким образом, можно подойти к гипотезе Пуанкаре и более общей 
проблеме 3-мерных сферических пространственных форм, задав следующий 
вопрос. С учетом соответствующих предположений о фундаментальной 
группе 3-мерного многообразия $M$, как установить существование 
метрики с постоянной кривизной Риччи на $M$? 
Основным элементом для создания такой метрики является уравнение 
потока Риччи, введённое Ричардом Гамильтоном в [29]:
\[
\frac{\partial g(t)}{\partial t} = -2Ric(g(t))
\]

\subsection{Основы геометрии Римана}

\textbf{2.1. Объем и радиус инъективности.} Одним важным общим понятием, которое 
используется в дальнейшем, является понятие того, что многообразие не схлопывается 
в какой-то точке. Пусть у нас есть точка $x$ в полном римановом $n$-мерном многообразии. 
Мы говорим, что многообразие \textit{$\kappa $-не-коллапсируемо} в точке $x$, если 
выполняется следующее условие: для любого $r$, при котором норма тензора кривизны Римана
$| Rm| $ не превышает $r^{-2}$ во всех точках метрического шара $B(x,r)$ радиуса $r$, центрированного 
в $x$, выполняется Vol$B(x,r)\geq \kappa r^{n}$. Существует связь между этим понятием и радиусом 
инъективности многообразия $M$ в точке $x$. А именно, если $| Rm| \leq r^{-2}$ на $B(x,r)$ и если 
$B(x,r)$ \textit{$\kappa $-не-коллапсируемо}, то радиус инъективности многообразия $M$ в точке 
$x$ больше или равен положительной константе, которая зависит только от $r$ и $\kappa $. 
Преимущество работы с условием объема, не приводящим к схлопыванию, заключается в том, 
что, в отличие от радиуса инъективности, существует простое уравнение для эволюции объема 
под действием потока Риччи.

Еще одним важным общим результатом является теорема о сравнении объемов Бишопа-Громова, 
которая утверждает, что если кривизна Риччи полного риманова $n$-мерного многообразия $M$ 
ограничена снизу постоянной $(n-1)K$, то для любой точки $x\in M$ отношение объема шара 
$B(x,r)$ к объему шара радиуса $r$ в пространстве постоянной кривизны $K$ является 
невозрастающей функцией, предел которой при $r\rightarrow 0$ равен $1$. 

Все эти базовые факты из римановой геометрии рассматриваются в первой главе.\\


\textbf{2.2. Многообразия с неотрицательной кривизной.} По причинам, которые должны стать 
ясными из вышеизложенного и, в любом случае, станут еще более очевидными вскоре, 
многообразия с неотрицательной кривизной играют чрезвычайно важную роль в анализе 
потоков Риччи с хирургией. Нам нужно несколько общих результатов о них. 
Первый — это теорема о душе для многообразий с неотрицательной секционной кривизной. 
\textit{Душа} — это компактное, тотально геодезическое подмногообразие. Все многообразие 
диффеоморфно полному пространству векторного расслоения над любой из своих душ. 
Если некомпактное $n$-мерное многообразие имеет положительную секционную кривизну, 
то любая душа для него — это точка, и, в частности, это многообразие диффеоморфно 
евклидову пространству. Кроме того, функция расстояния $f$ от души имеет свойство, 
что для каждого $t > 0$ прообраз $f^{-1} (t)$ гомеоморфен ($n-1$)-сфере, а прообраз под 
этой функцией расстояния любого недегенерированного интервала $I\subset \mathbb{R}^{+}$ гомеоморфен 
$S^{n-1} \times I$.

Другим важным результатом является теорема о расщеплении, которая гласит, что если полное 
многообразие неотрицательной кривизны в сечении имеет геодезическую линию (изометрическую копию 
$\mathbb{R}$), расстояние между каждой парой его точек которой минимизировано, то это многообразие 
является метрическим произведением многообразия на одну меньшую размерность и $\mathbb{R}$. В частности, если 
полное $n$-многообразие неотрицательной секционной кривизны имеет два конца, то это 
метрическое произведение $N^{n-1} \times \mathbb{R}$, где $N^{n-1}$ — компактное многообразие.

Кроме того, нам нужны некоторые элементарные результаты сравнения из теории Топоногова. 
В них сравниваются обычные треугольники на евклидовой плоскости с треугольниками в многообразии 
неотрицательной кривизны в разрезе, стороны которых минимизируют геодезические в этом 
многообразии.\\

\textbf{2.3. Канонические окрестности.} Большая часть анализа геометрии потоков Риччи связана 
с понятием канонических окрестностей. Зафиксируем некоторое достаточно малое значение 
$\epsilon > 0$. Существует два типа некомпактных канонических окрестностей: $\epsilon$-шейки и $\epsilon$-шапки.
В римановом 3-многообразии $(M,g)$ $\epsilon$-шея, центрированная в точке $x\in M$, — это подмногообразие 
$N\subset M$ и диффеоморфизм $\psi:S^{2} \times (-{\epsilon}^{-1} ,{\epsilon}^{-1})\rightarrow N$, 
такие что $x\in \psi(S^{2} \times \{0\})$, а также такие, 
что обратный образ масштабированной метрики $\psi^{*} (R(x)g)$ отличается от произведения 
стандартной метрики с кривизной $1$ на $S^{2}$ и обычной метрики на интервале $(-{\epsilon}^{-1} ,{\epsilon}^{-1})$ 
не более чем на $\epsilon$ в $C^{[1/\epsilon]}$-топологии. (Здесь $R(x)$ обозначает скалярную кривизну 
$(M,g)$ в точке $x$).
$\epsilon$-шапка — это некомпактное подмногообразие $\mathcal{C} \subset M$, обладающее свойством того, что 
некоторая окрестность бесконечности $N$ в $\mathcal{C}$ является $\epsilon$-шеей, а так же свойством того, что 
каждая точка $N$ является центром $\epsilon$-шеи в $M$ и что 
\textit{ядро} $\mathcal{C} \setminus \overline{N}$ $\epsilon$-шапки диффеоморфно либо 3-шару, либо проколотой $\mathbb{RP}^3$.
Также важно рассматривать $\epsilon$-шапки, которые после масштабирования 
(чтобы $R(x)=1$ для некоторой точки $x$ в шапке) обладают ограниченной 
геометрией (ограниченный диаметр, ограниченное соотношение кривизн 
в любых двух точках и ограниченный объём). Если $C$ задаёт границу 
для этих величин, то такую шапку называют $(C,\epsilon)$-шапкой (см. \textsc{Рис.} \ref{fig:fig1}).
$\epsilon$-трубка в $M$ — это подмногообразие $M$, диффеоморфное $S^{2}\times (0,1)$, 
являющееся объединением $\epsilon$-шеек и обладающее свойством, что каждая 
точка $\epsilon$-трубки является центром $\epsilon$-шеи в $M$.
\begin{figure}[h]
    \centering
    \includegraphics[width=\textwidth]{chapterIntro/subsection2/1.jpg}
    \caption{Канонические окрестности.}
    \label{fig:fig1}
\end{figure}

Существует два других типа канонических окрестностей в $3$-многообразиях: 
(i) $C$-компонента и (ii) $\epsilon$-круглая компонента. $C$-компонента — это компактное 
связное риманово многообразие с положительной секционной кривизной, 
диффеоморфное либо $S^{3}$, либо $\mathbb{RP}^3$, с тем свойством, что масштабирование 
метрики с использованием $R(x)$ для любой точки $x$ в компоненте приводит к 
римановому многообразию, диаметр которого не превышает $C$, секционная 
кривизна в любой точке и в любом направлении $2$-плоскости находится между 
$C^{-1}$ и $C$, а объём находится между $C^{-1}$ и $C$. 
$\epsilon$-круглая компонента — это компонента, на которой метрика, масштабированная 
с использованием $R(x)$ для любой точки $x$ в компоненте, находится в пределах 
$\epsilon$ в $C^{[1/\epsilon]}$-топологии от круглой метрики с скалярной кривизной, 
равной единице.

Как мы увидим, сингулярности на момент времени $T$ трехмерного 
потока Риччи содержатся в подмножествах, которые являются 
объединениями канонических окрестностей относительно метрик 
на более ранних моментах времени $t^{\prime } <T$. Таким образом, нам 
нужно понять топологию многообразий, которые являются 
объединениями $\epsilon$-трубок и $\epsilon$-шапок. Основное наблюдение состоит 
в том, что, при условии, что $\epsilon$ достаточно мало, когда две 
$\epsilon$-шейки пересекаются (в более чем небольшой окрестности их 
границ), их произведенные структуры почти выравниваются, так 
что две $\epsilon$-шейки можно склеить вместе, чтобы образовать 
многообразие, волоконированное $S^2$ -связкой над $S^1$. 
Этот топологический результат доказан в приложении в конце 
книги. Мы зафиксируем $\epsilon >0$, достаточно малое, чтобы эти 
результаты выполнялись. Используя эту идею, мы показываем, 
что для $\epsilon >0$, достаточно малого, если связное многообразие 
является объединением $\epsilon$-трубок и $\epsilon$-шапок, то оно 
диффеоморфно одному из следующих: $\mathbb{R}^3$, 
$S^2 \times \mathbb{R}$, $S^3$, $S^2 \times S^1$, $\mathbb{RP}^3 \#\mathbb{RP}^3$, 
полное пространство линейного расслоения над 
$\mathbb{RP}^2$ или неориентируемое $2$-сферическое расслоение над 
$S^1$. Этот топологический результат доказан в приложении в конце книги. 
\textbf{Мы зафиксируем $\epsilon >0$, достаточно малое, чтобы эти результаты выполнялись.}

Существует один результат, связанный с каноническими окрестностями 
и многообразиями с положительной кривизной, который мы будем 
использовать неоднократно: любое полное $3$-мерное многообразие с 
положительной кривизной не допускает $\epsilon$-шеек с произвольно высокой 
кривизной. В частности, если $M$ — полное риманово $3$-мерное 
многообразие с тем свойством, что каждая точка с скалярной 
кривизной больше $r^{-2}_{0}$ имеет каноническую окрестность, то 
$M$ имеет ограниченную кривизну. 
Это оказывается центральным результатом и используется многократно. 

Все эти базовые факты о римановых многообразиях с неотрицательной кривизной 
изложены во второй главе.


\subsection{Основы потока Риччи}

Гамильтон \cite{c29} ввёл уравнение потока Риччи.
\[
\frac{\partial g(t)}{\partial t} = -2Ric(g(t))
\]
Это уравнение эволюции для однопараметрического семейства 
римановых метрик $g(t)$ на гладком многообразии $M$. 
Уравнение потока Риччи является слабо параболическим и 
строго параболическим с учётом «группы калибровок», 
которая представляет собой группу диффеоморфизмов 
исходного гладкого многообразия. 
Это уравнение следует рассматривать как нелинейную, 
тензорную версию уравнения теплопроводности. 
Из него можно вывести уравнение эволюции для римановой метрики, 
тензора Риччи и функции скалярной кривизны. 
Все эти уравнения являются параболическими. 
Например, уравнение эволюции для скалярной кривизны 
$R(x,t)$ выглядит следующим образом:
\begin{equation}
    \frac{\partial R}{\partial t}(x,t) = \bigtriangleup R(x,t)+2{| Ric(x,t)|}^2
    \label{eqn0.1}
\end{equation}
иллюстрируя сходство с уравнением теплопроводности. 
(Здесь $\bigtriangleup$ — это лапласиан с неположительным спектром.)\\

\textbf{3.1. Первые результаты.} Конечно, первые результаты, которые нам нужны, 
— это уникальность и существование решений уравнения потока Риччи для 
компактных многообразий в короткий промежуток времени. Эти результаты 
были доказаны Гамильтоном (\cite{c29}) с использованием теоремы обратной функции 
Нэша-Мозера (\cite{c28}). Эти результаты являются стандартными для строго 
параболических уравнений. На данный момент существует достаточно стандартный 
метод работы «модуло» группы калибровок (группы диффеоморфизмов), что 
позволяет достичь строго параболической ситуации, в которой применяются 
классические результаты существования, уникальности и гладкости. Метод для 
уравнения потока Риччи называется «трюк де Тёрка».

Существует также результат, который позволяет сшить локальные решения 
$(U,g(t)),a\leq t\leq b$ и $(U,h(t)),b\leq t\leq c$ в одно гладкое решение, 
определённое на интервале $a\leq t\leq c$, при условии, что 
$g(b)=h(b)$.

Заданное уравнение потока Риччи $(M,g(t))$ можно всегда сдвинуть, 
заменив $t$ на $t+t_0$ для некоторого фиксированного $t_0$, что даст 
новый поток Риччи. Также можно масштабировать его на любое 
положительное число $Q$, задав $h(t)=Q g(Q^{-1} t)$, 
что создаст новый поток Риччи.\\

\textbf{3.2. Градиентные сжимающиеся солитоны.} Предположим, что 
$(M,g)$ — полное риманово многообразие, и существует константа 
$\lambda >0$, такая что:
\[
    Ric(g)=\lambda g
\]
В этом случае легко увидеть, что существует поток Риччи, задаваемый:
\[
    g(t)=(1-2\lambda t)g
\]
В частности, все метрики в этом потоке отличаются на постоянный множитель, 
зависящий от времени, и метрика является убывающей функцией времени. 
Такие решения называются \textit{сжимающимися солитонами}. Примеры включают 
компактные многообразия с постоянной положительной кривизной Риччи.

Существует тесно связанный с ними, но более общий класс примеров: 
\textit{градиентные сжимающиеся солитоны}. Предположим, что 
$(M,g)$ — полное риманово многообразие, и существует константа 
$\lambda >0$ и функция $f:M \rightarrow \mathbb{R}$, удовлетворяющие следующему условию:
\[
    Ric(g)=\lambda g - {Hess}^g f
\]
В данном случае существует поток Риччи, представляющий собой 
семейство сжимающихся метрик после применения обратного 
преобразования посредством одномерного семейства диффеоморфизмов, 
порожденного зависящим от времени векторным полем 
$\frac{1}{1-2\lambda t}{\bigtriangledown}_g f$. 
Примером градиентного убывающего солитона является многообразие 
$S^2 \times \mathbb{R}$ с семейством метрик, представляющим 
собой произведение семейства сжимающихся сферических метрик на 
$S^2$ и семейства постоянных стандартных метрик на $\mathbb{R}$. 
Функция $f$ в этом случае равна $s^2/4$, 
где $s$ — евклидова координата на $\mathbb{R}$.
\subsection{Решение потока гармонических отображений}
текст
\subsection{Решение потока гармонических отображений}
текст
\subsection{Решение потока гармонических отображений}
текст
\subsection{Решение потока гармонических отображений}
текст
\subsection{Решение потока гармонических отображений}
текст
\subsection{Список связанных статей}
текст


\newpage
\section{Введение}
\subsection{Метрика Римана и связность Леви-Чивиты}
текст
\subsection{Основы геометрии Римана}
и ещё текст
\subsection{Основы потока Риччи}
текст
\subsection{Вычисления в гауссовых нормальных координатах}
текст
\subsection{Основные результаты сравнения кривизны}
текст
\subsection{Локальный объем и радиус инъективности}
текст
\subsection{Исчезновение за конечное время}
текст
\subsection{Решение потока гармонических отображений}
текст
\subsection{Решение потока гармонических отображений}
текст


\newpage
\section{Многообразия с неотрицательной кривизной}
\subsection{Решение потока гармонических отображений}
текст
\subsection{Результаты сравнения в случае неотрицательной кривизны}
текст
\subsection{Теорема о душе}
текст
\subsection{Решение потока гармонических отображений}
текст
\subsection{Решение потока гармонических отображений}
текст
\subsection{\texorpdfstring{$\epsilon$-шейки}{epsilon-шейки}}
Текст о $\epsilon$-шейках.
\subsection{Коэффициенты прямой разницы}
текст
\newpage
\section{Название раздела нового}
\subsection{Определение потока Риччи}
текст
\subsection{Некоторые точные решения потока Риччи}
текст
\subsection{Локальная существуемость и единственность}
текст
\subsection{Решение потока гармонических отображений}
текст
\subsection{Эволюция кривизны в развивающейся ортонормальной системе}
текст
\subsection{Изменение расстояния под действием потока Риччи}
текст
\newpage
\section{Принцип максимума}
\subsection{Принцип максимума для скалярной кривизны}
текст
\subsection{Принцип максимума для тензоров}
текст
\subsection{Применения принципа максимума}
текст
\subsection{Сильный принцип максимума для кривизны}
текст
\subsection{Сужение к положительной кривизне}
текст

\newpage
\section{Результаты сходимости для потока Риччи}
\subsection{Геометрическая сходимость Римановых многообразий}
текст
\subsection{Геометрическая сходимость потоков Риччи}
текст
\subsection{Сходимость Громова–Хаусдорфа}
текст
\subsection{Пределы при увеличении масштаба}
текст
\subsection{Расщепление пределов на бесконечности}
текст

\newpage
\section{Геометрический подход к потоку Риччи через сравнения}
\subsection{\texorpdfstring{$\mathcal{L}$-длина и $\mathcal{L}$-геодезические}{L-длина и L-геодезические}}
Текст о $\mathcal{L}$-длинах и $\mathcal{L}$-геодезических

\subsection{\texorpdfstring{$\mathcal{L}$-экспоненциальное отображение и его свойства первого порядка}{L-экспоненциальное отображение и его свойства первого порядка}}
Текст о $\mathcal{L}$-экспоненциальном отображении и его свойствах первого порядка

\subsection{\texorpdfstring{Минимизирующие $\mathcal{L}$-геодезические и область инъективности}{Минимизирующие L-геодезические и область инъективности}}
Текст о минимизирующих $\mathcal{L}$-геодезических и области инъективности

\subsection{\texorpdfstring{Дифференциальные неравенства второго порядка для $\tilde{L}^{\overline{\tau}}$ и $L^{\overline{\tau}}_{x}$}{Дифференциальные неравенства второго порядка для Lτ и Lτx}}
Текст о дифференциальных неравенствах второго порядка для $\tilde{L}^{\overline{\tau}}$ и $L^{\overline{\tau}}_{x}$
\subsection{Решение потока гармонических отображений}
текст
\subsection{\texorpdfstring{Локальные оценки Липшица для $l_{x}$}{Локальные оценки Липшица для lx}}
Текст о локальных оценках Липшица для $l_{x}$

\subsection{Решение потока гармонических отображений}
текст
\newpage
\section{Название раздела нового}
\subsection{\texorpdfstring{Функции $L_{x}$ и $l_{x}$}{Функции Lx и lx}}
Текст о функциях $L_{x}$ и $l_{x}$
\subsection{\texorpdfstring{Оценка для $\min$ $l^{\tau}_{x}$}{Оценка для min lτx}}
Текст о оценке для $\min$ $l^{\tau}_{x}$

\subsection{Решение потока гармонических отображений}
текст
\subsection{Решение потока гармонических отображений}
текст
\subsection{Решение потока гармонических отображений}
текст
\subsection{Решение потока гармонических отображений}
текст
\newpage
\section{Название раздела нового}
\subsection{Результат о несхлопывании для обобщённых потоков Риччи}
текст
\subsection{Применение к компактным потокам Риччи}
текст
\subsection{Решение потока гармонических отображений}
текст
\subsection{Решение потока гармонических отображений}
текст
\subsection{Решение потока гармонических отображений}
текст
\subsection{Решение потока гармонических отображений}
текст
\newpage
\section{Название раздела нового}
\subsection{Решение потока гармонических отображений}
текст
\subsection{\texorpdfstring{Асимптотический градиентный сокращающий солитон для $\kappa$-решений}{Асимптотический градиентный сокращающий солитон для κ-решений}}
Асимптотический градиентный сокращающий солитон для $\kappa$-решений
\subsection{Решение потока гармонических отображений}
текст
\subsection{Классификация градиентных сокращающих солитонов в размерностях 2 и 3}
текст
\subsection{\texorpdfstring{Универсальный $\kappa$}{Универсальный κ}}
Универсальный $\kappa$
\subsection{Решение потока гармонических отображений}
текст
\newpage
\section{Название раздела нового}
\subsection{Сужение к положительному: определения}
текст
\subsection{Решение потока гармонических отображений}
текст
\subsection{Неполный геометрический предел}
текст
\subsection{\texorpdfstring{Пределы конуса возле конца $\mathcal{E}$ для рескейлингов $U_{\infty}$}{Пределы конуса возле конца E для рескейлингов U∞}}
Пределы конуса возле конца $\mathcal{E}$ для рескейлингов $U_{\infty}$

\subsection{Сравнение предела Громова–Хаусдорфа и гладкого предела}
текст
\subsection{Решение потока гармонических отображений}
текст
\newpage
\section{Название раздела нового}
\subsection{Гладкий предел при увеличении масштаба, определённый для малого времени}
текст
\subsection{Пределы при долгом времени увеличения масштаба}
текст
\subsection{Неполные гладкие пределы в сингулярные моменты}
текст
\subsection{\texorpdfstring{Существование сильных $\delta$-шеек, достаточно глубоких в $2\epsilon$-горне}{Существование сильных δ-шеек, достаточно глубоких в 2ϵ-горне}}
Существование сильных $\delta$-шеек, достаточно глубоких в $2\epsilon$-горне
\subsection{Решение потока гармонических отображений}
текст
\subsection{Решение потока гармонических отображений}
текст
\newpage
\section{Стандартное решение}
\subsection{Существование стандартного потока}
текст
\subsection{Полнота, положительная кривизна и асимптотическое поведение}
текст
\subsection{Стандартные решения являются вращательно симметричными}
текст
\subsection{Решение потока гармонических отображений}
текст
\subsection{Решение потока гармонических отображений}
текст
\subsection{Завершение доказательства единственности}
текст
\subsection{Решение потока гармонических отображений}
текст
\newpage
\section{\texorpdfstring{Хирургия на $\delta$-шейке}{Хирургия на δ-шейке}}
\subsection{Нотация и формулировка результата}
текст
\subsection{Решение потока гармонических отображений}
текст
\subsection{Доказательство теоремы 13.2}
текст
\subsection{Другие свойства результата хирургии}
текст
\newpage
\section{Название раздела нового}
\subsection{Пространство-время хирургии}
текст
\subsection{Обобщённое уравнение потока Риччи}
текст
\subsection{Решение потока гармонических отображений}
текст
\subsection{Решение потока гармонических отображений}
текст
\subsection{Решение потока гармонических отображений}
текст
\subsection{Решение потока гармонических отображений}
текст
\newpage
\section{Название раздела нового}
\subsection{Сшивание развивающихся шеек}
текст
\subsection{Топологические следствия предположений (1) – (7)}
текст
\subsection{Дополнительные условия для хирургии}
текст
\subsection{Решение потока гармонических отображений}
текст
\subsection{Утверждения о существовании потока Риччи с хирургией}
текст
\subsection{Контуры доказательства теоремы 15.9}
текст
\newpage
\section{Доказательство несхлопывания}
\subsection{Формулировка результата о несхлопывании}
текст
\subsection{\texorpdfstring{Доказательство несхлопывания при $R(x)=r^{-2}$ с $r\leq r_{i+1}$}{Доказательство несхлопывания при R(x) = r−2 с r ≤ ri+1}}

Доказательство несхлопывания при $R(x)=r^{-2}$ с $r \leq r_{i+1}$
\subsection{\texorpdfstring{Минимизирующие $\mathcal{L}$-геодезические существуют, когда $R(x)=r^{-2}_{i+1}$: формулировка}{Минимизирующие L-геодезические существуют, когда R(x)≤ri+1-2 : формулировка}}

Минимизирующие $\mathcal{L}$-геодезические существуют, когда $R(x)=r^{-2}_{i+1}$: формулировка
\subsection{Эволюция окрестностей хирургических кап}
текст
\subsection{Решение потока гармонических отображений}
текст
\subsection{Завершение доказательства пропозиции 16.1}
текст
\newpage
\section{Название раздела нового}
\subsection{Доказательство сильного предположения о канонических окрестностях}
текст
\subsection{Время хирургии не накапливается}
текст
\subsection{Решение потока гармонических отображений}
текст
\subsection{Решение потока гармонических отображений}
текст
\subsection{Решение потока гармонических отображений}
текст
\subsection{Решение потока гармонических отображений}
текст
\newpage
\section{Истечение до конечного времени}
\subsection{Решение потока гармонических отображений}
текст
\subsection{\texorpdfstring{Исчезновение компонентов с нетривиальной ${\pi}_{2}$}{Исчезновение компонентов с нетривиальной π2}}
Исчезновение компонентов с нетривиальной ${\pi}_{2}$
\subsection{Решение потока гармонических отображений}
текст
\subsection{Поток сжимающихся кривых}
текст
\subsection{Доказательство пропозиции 18.24}
текст
\subsection{Доказательство леммы 18.59: кольца с маленькой площадью}
текст
\subsection{Доказательство первой неравенства в лемме 18.52}
текст
\newpage
\section{Приложение: Канонические окрестности}
\subsection{Решение потока гармонических отображений}
текст
\subsection{\texorpdfstring{Геометрия $\epsilon$-шеек}{Геометрия ϵ-шеек}}
Геометрия $\epsilon$-шеек
\subsection{\texorpdfstring{Перекрывающиеся $\epsilon$-шейки}{Перекрывающиеся ϵ-шейки}}
Перекрывающиеся $\epsilon$-шейки
\subsection{\texorpdfstring{Области, покрытые $\epsilon$-шейками и $(C,\epsilon)$-капами}{Области, покрытые ϵ-шейками и (C,ϵ)-капами}}
Области, покрытые $\epsilon$-шейками и $(C,\epsilon)$-капами
\subsection{\texorpdfstring{Подмножества объединения ядер $(C,\epsilon)$-кап и $\epsilon$-шеек}{Подмножества объединения ядер (C,ϵ)-капов и ϵ-шеек}}
Подмножества объединения ядер $(C,\epsilon)$-капов и $\epsilon$-шеек
\newpage

\addcontentsline{toc}{section}{Список литературы}
\begin{thebibliography}{99}
    \bibitem{c1} 
    Steven Altschuler. Singularities of the curve shrinking flow for space curves. \textit{J. Differential Geometry}, 34:491–514, 1991.
    
    \bibitem{c2}
    Steven Altschuler and Matthew Grayson. Shortening space curves and flow through singularities. \textit{J. Differential Geom.}, 35:283–298, 1992.
    
    \bibitem{c3}
    Shigetoshi Bando. Real analyticity of solutions of Hamilton’s equation. \textit{Math. Z.}, 195(1):93–97, 1987.
    
    \bibitem{c4}
    Yu. Burago, M. Gromov, and G. Perel\`man. A. D. Aleksandrov spaces with curvatures bounded below. \textit{Uspekhi Mat. Nauk}, 47(2(284)):3–51, 222, 1992.
    
    \bibitem{c5}
    Huai-Dong Cao and Xi-Ping Zhu. A complete proof of the Poincaré and Geometrization conjectures – Application of the Hamilton-Perelman theory of the Ricci flow. \textit{Asian J. of Math}, 10:169–492, 2006.
    
    \bibitem{c6}
    Jeff Cheeger. Finiteness theorems for Riemannian manifolds. \textit{Amer. J. Math.}, 92:61–74, 1970.
    
    \bibitem{c7}
    Jeff Cheeger and David Ebin. Comparison theorems in Riemannian geometry. \textit{North-Holland Publishing Co.}, Amsterdam, 1975. North-Holland Mathematical Library, Vol. 9.
    
    \bibitem{c8}
    Jeff Cheeger and Detlef Gromoll. The structure of complete manifolds of nonnegative curvature. \textit{Bull. Amer. Math. Soc.}, 74:1147–1150, 1968.
    
    \bibitem{c9}
    Jeff Cheeger and Detlef Gromoll. The splitting theorem for manifolds of nonnegative Ricci curvature. \textit{J. Differential Geometry}, 6:119–128, 1971/1972.
    
    \bibitem{c10}
    Jeff Cheeger and Detlef Gromoll. On the structure of complete manifolds of nonnegative curvature. \textit{Ann. of Math. (2)}, 96:413–433, 1972.
    
    \bibitem{c11}
    Jeff Cheeger, Mikhail Gromov, and Michael Taylor. Finite propagation speed, kernel estimates for functions of the Laplace operator, and the geometry of complete Riemannian manifolds. \textit{J. Differential Geom.}, 17(1):15–53, 1982.
    
    \bibitem{c12}
    B. L. Chen and X. P. Zhu. Uniqueness of the Ricci flow on complete noncompact manifolds. \textit{math.DG/0505447}, 2005.
    
    \bibitem{c13}
    Bennett Chow and Dan Knopf. The Ricci flow: an introduction, volume 110 of Mathematical Surveys and Monographs. \textit{American Mathematical Society}, Providence, RI, 2004.
    
    \bibitem{c14}
    Bennett Chow, Peng Lu, and Li Ni. Hamilton\'s Ricci flow. to appear, 2006.
    
    \bibitem{c15}
    Tobias H. Colding and William P. Minicozzi, II. Estimates for the extinction time for the Ricci flow on certain 3-manifolds and a question of Perelman. \textit{J. Amer. Math. Soc.}, 18(3):561–569 (electronic), 2005.
    
    \bibitem{c16}
    Dennis M. DeTurck. Deforming metrics in the direction of their Ricci tensors. \textit{J. Differential Geom.}, 18(1):157–162, 1983.
    
    \bibitem{c17}
    Yu Ding. Notes on Perelman\'s second paper. Available at \texttt{www.math.lsa.umich.edu/\~{}lott/ricciflow/perelman.html}, 2004.
    
    \bibitem{c18}
    M. P. do Carmo. Riemannian Geometry. \textit{Birkhäuser}, Boston, 1993.
    
    \bibitem{c19}
    Klaus Ecker and Gerhard Huisken. Interior estimates for hypersurfaces moving by mean curvature. \textit{Invent. Math.}, 105:547–569, 1991.
    
    \bibitem{c20}
    Lawrence C. Evans and Ronald F. Gariepy. Measure theory and fine properties of functions. \textit{Studies in Advanced Mathematics. CRC Press}, Boca Raton, FL, 1992.
    
    \bibitem{c21}
    M. Gage and R. S. Hamilton. The heat equation shrinking convex plane curves. \textit{J. Differential Geom.}, 23(1):69–96, 1986.
    
    \bibitem{c22}
    Sylvestre Gallot, Dominique Hulin, and Jacques Lafontaine. \textit{Riemannian geometry}. Universitext. Springer-Verlag, Berlin, third edition, 2004.

    \bibitem{c23}
    R. E. Greene and H. Wu. Lipschitz convergence of Riemannian manifolds. \textit{Pacific J. Math.}, 131:119–141, 1988.

    \bibitem{c24}
    Detlef Gromoll and Wolfgang Meyer. On complete open manifolds of positive curvature. \textit{Ann. of Math. (2)}, 90:75–90, 1969.

    \bibitem{c25}
    Mikhael Gromov. Structures métriques pour les variétés riemanniennes, volume 1 of Textes mathématiques. CEDIC/Fernand Nathan, Paris, France, 1981.

    \bibitem{c26}
    Mikhael Gromov and H. Blaine Lawson, Jr. Positive scalar curvature and the Dirac operator on complete Riemannian manifolds. \textit{Inst. Hautes Etudes Sci. Publ. Math.}, 58:83–196, 1983.

    \bibitem{c27}
    Robert Gulliver and Frank David Lesley. On boundary branch points of minimizing surfaces. \textit{Arch. Rational Mech. Anal.}, 52:20–25, 1973.

    \bibitem{c28}
    Richard S. Hamilton. The inverse function theorem of Nash and Moser. \textit{Bull. Amer. Math. Soc. (N.S.)}, 7(1):65–222, 1982.

    \bibitem{c29}
    Richard S. Hamilton. Three-manifolds with positive Ricci curvature. \textit{J. Differential Geom.}, 17(2):255–306, 1982.

    \bibitem{c30}
    Richard S. Hamilton. Four-manifolds with positive curvature operator. \textit{J. Differential Geom.}, 24(2):153–179, 1986.

    \bibitem{c31}
    Richard S. Hamilton. The Ricci flow on surfaces. In \textit{Mathematics and general relativity} (Santa Cruz, CA, 1986), volume 71 of Contemp. Math., pages 237–262. Amer. Math. Soc., Providence, RI, 1988.

    \bibitem{c32}
    Richard S. Hamilton. The Harnack estimate for the Ricci flow. \textit{J. Differential Geom.}, 37(1):225–243, 1993.

    \bibitem{c33}
    Richard S. Hamilton. A compactness property for solutions of the Ricci flow. \textit{Amer. J. Math.}, 117(3):545–572, 1995.

    \bibitem{c34}
    Richard S. Hamilton. The formation of singularities in the Ricci flow. In \textit{Surveys in differential geometry, Vol. II} (Cambridge, MA, 1993), pages 7–136. Internat. Press, Cambridge, MA, 1995.

    \bibitem{c35}
    Richard S. Hamilton. Four-manifolds with positive isotropic curvature. \textit{Comm. Anal. Geom.}, 5(1):1–92, 1997.

    \bibitem{c36}
    Richard S. Hamilton. Non-singular solutions of the Ricci flow on three-manifolds. \textit{Comm. Anal. Geom.}, 7(4):695–729, 1999.

    \bibitem{c37}
    Robert Hardt and Leon Simon. Boundary regularity and embedded solutions for the oriented Plateau problem. \textit{Ann. of Math. (2)}, 110(3):439–486, 1979.

    \bibitem{c38}
    Allen Hatcher. \textit{Algebraic topology}. Cambridge University Press, Cambridge, 2002.

    \bibitem{c39}
    John Hempel. \textit{3-Manifolds}. Princeton University Press, Princeton, N. J., 1976. Ann. of Math. Studies, No. 86.

    \bibitem{c40}
    Stefan Hildebrandt. Boundary behavior of minimal surfaces. \textit{Arch. Rational Mech. Anal.}, 35:47–82, 1969.

    \bibitem{c41}
    T. Ivey. Ricci solitons on compact three-manifolds. \textit{Diff. Geom. Appl.}, 3:301–307, 1993.

    \bibitem{c42}
    Jürgen Jost. \textit{Two-dimensional geometric variational problems}. Pure and Applied Mathematics (New York). John Wiley \& Sons Ltd., Chichester, 1991. A Wiley-Interscience Publication.

    \bibitem{c43}
    Vatali Kapovitch. Perelman\`s stability theorem. \textit{math.DG/0703002}, 2007.

    \bibitem{c44}
    Bruce Kleiner and John Lott. Notes on Perelman\`s papers. \textit{math.DG/0605667}, 2006.

    \bibitem{c45}
    O. A. Ladyzhenskaja, V. A. Solonnikov, and N. N. Ural\`ceva. \textit{Linear and quasilinear equations of parabolic type}. Translated from the Russian by S. Smith. Translations of Mathematical Monographs, Vol. 23. American Mathematical Society, Providence, R.I., 1967.

    \bibitem{c46}
    Peter Li and L-F. Tam. The heat equation and harmonic maps of complete manifolds. \textit{Invent. Math.}, 105:305–320, 1991.

    \bibitem{c47}
    Peter Li and Shing-Tung Yau. On the parabolic kernel of the Schrödinger operator. \textit{Acta Math.}, 156(3-4):153–201, 1986.

    \bibitem{c48}
    Peng Lu and Gang Tian. Uniqueness of standard solutions in the work of Perelman. Available at \texttt{www.math.lsa.umich.edu/\~lott/ricciflow/StanUniqWork2.pdf}, 2005.

    \bibitem{c49}
    John Milnor. Towards the Poincaré conjecture and the classification of 3-manifolds. \textit{Notices Amer. Math. Soc.}, 50(10):1226–1233, 2003.

    \bibitem{c50}
    J. Morgan and G. Tian. Completion of Perelman\'s proof of the Geometrization Conjecture. In preparation, 2007.

    \bibitem{c51}
    C. B. Morrey. The problem of Plateau on a Riemannian manifold. \textit{Ann. Math.}, 49:807–851, 1948.

    \bibitem{c52}
    G. Perelman. The entropy formula for the Ricci flow and its geometric applications. math.DG/0211159, 2002.

    \bibitem{c53}
    G. Perelman. Finite extinction time for the solutions to the Ricci flow on certain three-manifolds. math.DG/0307245, 2003.

    \bibitem{c54}
    G. Perelman. Ricci flow with surgery on three-manifolds. math.DG/0303109, 2003.

    \bibitem{c55}
    S. Peters. Convergence of Riemannian manifolds. \textit{Compositio Math.}, 62:3–16, 1987.

    \bibitem{c56}
    P. Petersen. \textit{Riemannian geometry, Second Edition}, volume 171 of Graduate Texts in Mathematics. Springer-Verlag, New York, 2006.

    \bibitem{c57}
    H. Poincaré. Cinquième complément à l\'analyse situs. In \textit{Œuvres. Tome VI, Les Grands Classiques Gauthier-Villars}, pages v+541. Editions Jacques Gabay, Sceaux, 1996. Reprint of the 1953 edition.

    \bibitem{c58}
    J. Sacks and K. Uhlenbeck. The existence of minimal immersions of 2-spheres. \textit{Ann. of Math.}, 113:1–24, 1981.

    \bibitem{c59}
    T. Sakai. \textit{Riemannian geometry}, volume 149 of Translations of Mathematical Monographs. American Mathematical Society, Providence, RI, 1996. Translated from the 1992 Japanese original by the author.

    \bibitem{c60}
    R. Schoen and S.-T. Yau. \textit{Lectures on differential geometry}. Conference Proceedings and Lecture Notes in Geometry and Topology, I. International Press, Cambridge, MA, 1994.

    \bibitem{c61}
    R. Schoen and S.-T. Yau. The structure of manifolds with positive scalar curvature. In \textit{Directions in partial differential equations (Madison, WI, 1985)}, volume 54 of Publ. Math. Res. Center Univ. Wisconsin, pages 235–242. Academic Press, Boston, MA, 1987.

    \bibitem{c62}
    P. Scott. The geometries of 3-manifolds. \textit{Bull. London Math. Soc.}, 15(5):401–487, 1983.

    \bibitem{c63}
    N. Sesum, G. Tian, and X.-D. Wang. Notes on Perelman\'s paper on the entropy formula for the Ricci flow and its geometric applications. preprint, 2003.

    \bibitem{c64}
    W.-X. Shi. Deforming the metric on complete Riemannian manifolds. \textit{J. Differential Geom.}, 30(1):223–301, 1989.

    \bibitem{c65}
    W.-X. Shi. Ricci deformation of the metric on complete noncompact Riemannian manifolds. \textit{J. Differential Geom.}, 30(2):303–394, 1989.

    \bibitem{c66}
    T. Shioya and T. Yamaguchi. Volume collapsed three-manifolds with a lower curvature bound. \textit{Math. Ann.}, 333(1):131–155, 2005.

    \bibitem{c67}
    J. Stallings. A topological proof of Gruschko\'s theorem on free products. \textit{Math. Z.}, 90:1–8, 1965.

    \bibitem{c68}
    W. P. Thurston. Hyperbolic structures on 3-manifolds. I. Deformation of acylindrical manifolds. \textit{Ann. of Math. (2)}, 124(2):203–246, 1986.

    \bibitem{c69}
    V. Toponogov. Spaces with straight lines. \textit{AMS Translations}, 37:287–290, 1964.

    \bibitem{c70}
    B. White. Classical area minimizing surfaces with real-analytic boundaries. \textit{Acta Math.}, 179(2):295–305, 1997.

    \bibitem{c71}
    R. Ye. On the l function and the reduced volume of Perelman. Available at www.math.ucsb.edu/\~yer/reduced.pdf, 2004.

\end{thebibliography}

\end{document}