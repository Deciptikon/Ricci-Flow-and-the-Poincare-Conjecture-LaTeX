\documentclass{article}
\usepackage{amsthm}
\usepackage[utf8]{inputenc} % Кодировка UTF-8
\usepackage[T2A]{fontenc}   % Кодировка для кириллицы
\usepackage[russian]{babel} % Поддержка русского языка
\usepackage[colorlinks=true,linkcolor=blue]{hyperref}
\usepackage{tocloft} % Для настройки оглавления
\usepackage{footnotehyper}

% Убираем точки для всех уровней содержания
\renewcommand{\cftsecleader}{\cftdotfill{\cftnodots}} % Для разделов
\renewcommand{\cftsubsecleader}{\cftdotfill{\cftnodots}} % Для подразделов (если есть)

% Настраиваем нумерацию в содержании
\renewcommand{\thesubsection}{\arabic{subsection}}
\renewcommand{\thesection}{Глава \arabic{section}.}
\renewcommand{\cftsecaftersnum}{ }
\cftsetindents{section}{0em}{5.0em}

% Настройка вида блоков

% Или для другого шрифта, например, шрифт без засечек:
\newtheoremstyle{theoremstyle}
{10pt}         % Вершина
{10pt}         % Низ
{\normalfont}  % Обычный шрифт для текста заголовка
{}             % Отступ
{\scshape}     % Small Caps для заголовка
{.}            % Печать
{10pt}          % Разделитель
{}             % Стиль

\theoremstyle{theoremstyle}
\newtheorem{theorem}{Теорема}[section] % Нумерация Теорем по главам
\renewcommand{\thetheorem}{\arabic{section}.\arabic{theorem}}

\theoremstyle{theoremstyle}
\newtheorem{corollary}[theorem]{Следствие} % Нумерация Следствий по главам
\renewcommand{\thecorollary}{\arabic{section}.\arabic{corollary}}

\theoremstyle{theoremstyle}
\newtheorem{remark}[theorem]{Замечание} % Нумерация Замечаний по главам
\renewcommand{\theremark}{\arabic{section}.\arabic{remark}}

\theoremstyle{theoremstyle}
\newtheorem*{myproof}{Доказательство}

\begin{document}

\begin{center}
    \huge\textbf{Поток Риччи и гипотеза Пуанкаре}
    \vspace{1cm}

    \huge{Джон В. Морган и Ганг Тянь}
\end{center}
\newpage
\newpage
\tableofcontents
\newpage
\section*{Введение}
\addcontentsline{toc}{section}{Введение}

В этой книге представлено полное и детализированное доказательство 

\hspace{1cm}\large\textbf{гипотезы Пуанкаре: всякое замкнутое, гладкое}

\hspace{1cm}\large\textbf{односвязное 3-многообразие диффеоморфно}\footnote{Каждое топологическое 3-мерное многообразие обладает дифференцируемой структурой, и каждый гомеоморфизм между гладкими 3-мерными многообразиями может быть сведён к диффеоморфизму. Таким образом, результаты классификации топологических 3-мерных многообразий до гомеоморфизма и гладких 3-мерных многообразий до диффеоморфизма эквивалентны. В этой книге под «многообразием» подразумевается «гладкое многообразие».} \large\textbf{$S^{3}$}\\
Эта гипотеза была сформулирована Анри Пуанкаре \cite{c58} в 1904 году и оставалась открытой 
до недавней работы Перельмана. Аргументы, представленные здесь, являются детализированной 
версией тех, которые встречаются в трех препринтах Перельмана \cite{c53, c55, c54}. 
Аргументы Перельмана опираются на фундамент, созданный Ричардом Гамильтоном, 
с его исследованием уравнения потока Риччи для Римановых метрик. 
На самом деле, Гамильтон считал, что потоки Риччи можно использовать 
для установления гипотезы Пуанкаре и более общих результатов топологической 
классификации в размерности 3, и разработал программу для достижения этой цели. 
Трудность заключалась в том, чтобы справиться с особенностями в потоках Риччи. 
Прорыв Перельмана заключался в том, чтобы понять качественную природу 
особенностей достаточно хорошо, чтобы позволить ему доказать гипотезу Пуанкаре 
(и теорему 0.1 ниже, которая подразумевает гипотезу Пуанкаре). 
Для подробной истории гипотезы Пуанкаре см. обзорную статью Мильнора \cite{c50}.\\

Класс примеров, тесно связанных с 3-сферой, составляют 3-\textit{мерные сферические 
пространственные формы}, т.е. отношения $S^{3}$ по свободным линейным действиям 
конечных подгрупп ортогональной группы $O(4)$. 
Существует обобщение гипотезы Пуанкаре, называемое \textbf{гипотезой} 3-\textbf{мерных 
сферических пространственных форм}, которое утверждает, что любое замкнутое 
3-мерное многообразие с конечной фундаментальной группой диффеоморфно 
3-мерной сферической пространственной форме. Очевидно, что частным случаем 
гипотезы 3-мерных сферических пространственных форм является гипотеза Пуанкаре.

Как указано в замечании 1.4 в \cite{c54}, аргументы, представленные здесь, не только 
доказывают гипотезу Пуанкаре, но и доказывают гипотезу 3-мерных пространственных 
форм. На самом деле, цель этой книги — доказать следующую более общую теорему.\vspace{0.5em}

Теорема 0.1. \textit{Пусть $M$ — замкнутое, связное 3-мерное многообразие, 
и пусть фундаментальная группа $M$ является свободным произведением конечных 
групп и бесконечных циклических групп. Тогда $M$ диффеоморфно связной сумме 
сферических пространственных форм, копий $S^{2}\times S^{1}$ и копий 
уникального (до диффеоморфизма) неориентируемого 2-сферического расслоения над 
$S^{1}$.}\vspace{0.5em}

Это немедленно означает положительное разрешение гипотезы Пуанкаре и гипотезы 
3-мерных сферических пространственных форм.\vspace{0.5em}

Следствие 0.2.\textit{
(a) Замкнутое, просто связное 3-мерное многообразие диффеоморфно $S^{3}$.
(b) Замкнутое 3-мерное многообразие с конечной фундаментальной группой диффеоморфно 
3-мерной сферической пространственной форме.}\vspace{0.5em}

Прежде чем перейти к более подробному описанию содержания этой книги, стоит 
сделать одно замечание относительно стиля изложения. Из-за важности и 
заметности обсуждаемых результатов, а также из-за множества неверных утверждений 
о доказательствах этих результатов в прошлом, мы посчитали необходимым 
представить аргументы с большой детализацией. Наша цель состояла в том, чтобы 
сделать эти аргументы ясными и убедительными, а также более доступными для 
широкой аудитории. В результате эксперты могут посчитать некоторые моменты 
излишне подробно изложенными.

\subsection{Обзор аргументов Перельмана}
В размерностях, меньших или равных трём, любая риманова метрика с 
постоянной кривизной Риччи имеет постоянную секционную кривизну. 
Классические результаты в римановой геометрии показывают, что 
универсальное покрытие замкнутого многообразия с постоянной 
положительной кривизной диффеоморфно сфере, а фундаментальная 
группа идентифицируется с конечной подгруппой ортогональной группы, 
которая действует линейно и свободно на универсальном покрытии. 
Таким образом, можно подойти к гипотезе Пуанкаре и более общей 
проблеме 3-мерных сферических пространственных форм, задав следующий 
вопрос. С учетом соответствующих предположений о фундаментальной 
группе 3-мерного многообразия $M$, как установить существование 
метрики с постоянной кривизной Риччи на $M$? 
Основным элементом для создания такой метрики является уравнение 
потока Риччи, введённое Ричардом Гамильтоном в \cite{c29}:
\[
\frac{\partial g(t)}{\partial t} = -2Ric(g(t))
\]
Где $Ric(g(t))$ — это кривизна Риччи метрики $g(t)$. Фиксированные точки 
(с точностью до масштабирования) этого уравнения — римановы метрики с 
постоянной кривизной Риччи. Для общего введения в тему потоков Риччи 
см. обзорную статью Гамильтона \cite{c34}, книгу Чоу и Кнопфа \cite{c13} или 
книгу Чоу, Лу и Ни \cite{c14}.
Уравнение потока Риччи — это (слабо) параболическое уравнение в 
частных производных для римановых метрик на гладком многообразии. 
Следуя Гамильтону, поток Риччи определяется как семейство римановых 
метрик $g(t)$ на фиксированном гладком многообразии, параметризованное параметром 
$t$ на некотором интервале, которое удовлетворяет этому уравнению. 
Параметр $t$ интерпретируется как время, а уравнение рассматривается 
как задача Коши: начиная с любого риманова многообразия $(M, g_{0})$, 
требуется найти поток Риччи с начальной метрикой $g_{0}$; то есть найти 
однопараметрическое семейство $(M, g(t))$ римановых многообразий, 
где $g(0) = g_{0}$, удовлетворяющее уравнению потока Риччи. 
Это уравнение справедливо в любых размерностях, однако здесь мы 
сосредотачиваемся на трёхмерном случае.
Вкратце, метод доказательства заключается в том, чтобы начать с 
любой римановой метрики на заданном гладком 3-мерном многообразии 
и применить к ней поток Риччи, чтобы получить метрику с постоянной 
кривизной, которую мы ищем. Есть два примера, где этот метод 
работает именно так, оба принадлежат Гамильтону.\\
(i) Если начальная метрика имеет положительную кривизну Риччи, 
Гамильтон доказал более 20 лет назад (\cite{c29}), что при потоке Риччи 
многообразие сжимается до точки за конечное время, то есть возникает 
сингулярность в конечное время, и, по мере приближения к сингулярности, 
диаметр многообразия стремится к нулю, а кривизна стремится к 
бесконечности в каждой точке. Гамильтон также показал, что в этом 
случае скейлинг метрики с помощью зависящей от времени функции, 
так чтобы диаметр оставался постоянным, приводит к однопараметрическому 
семейству метрик, которые гладко сходятся к метрике с постоянной 
положительной кривизной.\\
(ii) На другом крайнем случае, в работе \cite{c36} Гамильтон показал, что если 
поток Риччи существует для любого времени и если выполняются 
соответствующие ограничения на кривизну, а также другое геометрическое 
ограничение, то при $t \rightarrow \infty$, после скейлинга для фиксированного 
диаметра, метрика сходится к метрике с постоянной отрицательной кривизной.

Результаты в общем случае значительно сложнее для формулировки и намного 
труднее для доказательства. Хотя Гамильтон установил, что уравнение потока 
Риччи обладает свойствами краткосрочного существования, то есть можно определить 
$g(t)$ для $t$ на интервале $[0,T)$, где $T$ зависит от начальной метрики, оказывается, что 
если топология многообразия достаточно сложна, например, оно представляет собой 
нетривиальную связанную сумму, то независимо от начальной метрики неизбежно 
возникнут сингулярности за конечное время, вызванные топологией. Более того, 
даже если многообразие имеет простую топологию, начиная с произвольной метрики, 
можно ожидать (и нельзя исключить возможность) возникновения сингулярностей за 
конечное время в потоке Риччи. Эти сингулярности, в отличие от случая положительной 
кривизны Риччи, появляются на правильных подмножествах многообразия, 
а не на всем многообразии.
Таким образом, для вывода топологических последствий, указанных выше, в общем 
случае недостаточно остановить анализ в момент появления первой сингулярности 
в потоке Риччи. Это приводит к изучению более общего процесса эволюции, 
называемого \textit{потоком Риччи с хирургией}, впервые введенного Гамильтоном в контексте 
четырехмерных многообразий. Этот процесс эволюции также параметризован интервалом 
времени, так что для каждого $t$ из интервала определения существует компактное 
риманово трехмерное многообразие $M_{t}$. Однако существует дискретный набор моментов 
времени, в которых многообразия и метрики претерпевают топологические и 
метрические разрывы (хирургии). В каждом из интервалов между такими моментами 
эволюция подчиняется обычному потоку Риччи, но из-за хирургии топологический 
тип многообразия $M_{t}$ меняется при переходе из одного интервала в следующий.
С аналитической точки зрения хирургии в моменты разрывов вводятся для того, 
чтобы `вырезать' окрестности сингулярностей по мере их развития и вручную 
вставить вместо `вырезанных'  областей геометрически правильные регионы. 
Это позволяет продолжить (точнее, перезапустить) поток Риччи с новой метрикой, 
построенной в момент разрыва. Конечно, процесс хирургии также изменяет топологию. 
Чтобы можно было сделать полезные топологические выводы о таком процессе, 
необходимо как иметь результаты о потоке Риччи, так и контролировать топологию и 
геометрию процесса хирургии в моменты сингулярностей. Например, важно для 
топологических приложений, чтобы хирургии выполнялись вдоль 2-сфер, а не 
поверхностей более высокого рода. Хирургия вдоль 2-сфер создает разложение на 
связанные суммы, которое хорошо изучено в топологии, в то время как, например, 
операции Дена вдоль торов могут полностью разрушить топологию, превращая любое 
трехмерное многообразие в любое другое.

Изменение топологии оказывается полностью понятным, и, что удивительно, 
процессы хирургии создают именно те топологические операции, которые необходимы, 
чтобы разрезать многообразие на части, где поток Риччи может сформировать 
метрики, достаточно контролируемые для распознавания топологии.

Основная часть этой книги (главы 1—17 и приложение) посвящена установлению 
следующего результата о долгосрочном существовании потока Риччи с хирургией.

\begin{theorem}
\textit{Пусть $(M, g_{0})$ — замкнутое риманово трехмерное 
многообразие. Предположим, что в $M$ нет вложенных, локально разделенных областей\footnote{Т.е. в $M$ нет вложенных $RP^{2}$ с тривиальным нормальным расслоением. Очевидно, все ориентируемые многообразия удовлетворяют этому условию.} 
$\mathbb{RP}^2$. Тогда существует поток Риччи с хирургией, 
определенный для всех $t\in [0,\infty)$ с начальной метрикой $(M, g_{0})$. 
Множество моментов разрывов для этого потока Риччи с хирургией является 
дискретным подмножеством $[0,\infty)$.
Топологическое изменение трехмерного многообразия при прохождении через 
момент хирургии представляет собой разложение на связанную сумму с 
удалением связных компонент, каждая из которых диффеоморфна одному из 
следующих многообразий: $S^{2}\times S^{1}, \mathbb{RP}^3 \# \mathbb{RP}^3$, не-ориентируемое 
расслоение над $S^{1}$ с базой $S^{2}$, или многообразие, допускающее 
метрику с постоянной положительной кривизной.}
\end{theorem}

Хотя Теорема 0.3 является ключевой для всех приложений потока Риччи к 
топологии трёхмерных многообразий, доказательство для трёхмерных многообразий, 
описанных в Теореме 0.1, упрощено и избегает любых ссылок на поведение потока 
при стремлении времени к бесконечности благодаря следующему результату о 
конечновременном вырождении.

\begin{theorem}
\textit{Пусть $M$ — замкнутое 3-многообразие, фундаментальная группа которого 
является свободным произведением конечных групп и бесконечных циклических групп\footnote{В \cite{c54} Перельман утверждает результат для 3-многообразий без простых факторов (неделимых многообразий), которые являются ацикличными. Это стандартное упражнение в топологии 3-многообразий — показать, что условие Перельмана эквивалентно гипотезе из теории групп, изложенной здесь; см. Следствие 0.5.}. 
Пусть $g_{0}$ — произвольная риманова метрика на $M$. Тогда $M$ не содержит локально 
разделяющихся $\mathbb{RP}^2$, так что поток Риччи с хирургией, определённый для любого 
положительного времени, может быть задан, начиная с метрики $(M,g_{0})$, как описано 
в Теореме 0.3. Этот поток Риччи с хирургией завершает своё существование через 
конечное время $T<\infty$ в том смысле, что многообразия $M_{t}$ пусты для всех $t \geq T$.}
\end{theorem}

Этот результат устанавливается в главе 18, следуя доводу, предложенному Перельманом 
в \cite{c54}, см. также \cite{c15}.

Мы можем вывести Теорему 0.1 напрямую из Теорем 0.3 и 0.4 следующим образом:
Пусть $M$ — 3-многообразие, удовлетворяющее гипотезе Теоремы 0.1. Тогда 
существует конечная последовательность $M=M_{0},M_{1},\ldots ,M_{k}=\emptyset $, такая, что для 
каждого $i$, $1\leq i\leq k$, $M_{i}$ получается из $M_{i-1}$ либо с помощью разложения на связную 
сумму, либо путём удаления компоненты, диффеоморфной одной из следующих: 
$S^{2}\times S^{1}$ , $\mathbb{RP}^3 \# \mathbb{RP}^3$ , неориентируемого расслоения с базой $S^{1}$ и слоем 
$S^{2}$, или трёхмерной сферической пространственной формы.

Очевидно, по нисходящей индукции по $i$ следует, что каждая связная компонента 
$M_{i}$ диффеоморфна связной сумме трёхмерных сферических пространственных форм, 
копий $S^{2} \times S^{1}$ и копий неориентируемого расслоения с базой $S^{1}$ и слоем 
$S^{2}$. В частности, $M=M_{0}$ имеет эту форму. Так как $M$ предположительно связно, 
это доказывает теорему. Более того, этот довод доказывает следующее:

\begin{corollary}
\textit{Пусть $(M_{0},g_{0})$ — связное риманово многообразие, не содержащее 
локально разделяющихся $\mathbb{RP}^2$. Пусть $(M,G)$ — поток Риччи с хирургией, определённый для 
$0\leq t<\infty$, с $(M_{0},g_{0})$ в качестве начального многообразия. Тогда следующие четыре условия 
эквивалентны:
\begin{enumerate}
    \item $(M,G)$ прекращает своё существование за конечное время, то есть $M_{T}=\emptyset $ для всех достаточно 
    больших $T$. 
    \item $M_{0}$ диффеоморфно связной сумме трёхмерных сферических пространственных форм и расслоений 
    $S^{2}$ над $S^{1}$. 
    \item Фундаментальная группа $M_{0}$ является свободным произведением конечных групп и 
    бесконечных циклических групп. 
    \item Ни один простой\footnote{Трёхмерное многообразие $P$ называется простым, если любая 
    разделяющая двумерная сфера в $P$ ограничивает трёхмерный шар в $P$. Эквивалентно, $P$ является 
    простым, если не допускает нетривиального разложения в виде связной суммы. Каждое замкнутое 
    трёхмерное многообразие раскладывается в связную сумму простых факторов, причём это разложение 
    единственно с точностью до диффеоморфизма и порядка факторов.} 
    фактор $M_{0}$ не является ацикличным, то есть 
    каждый простой фактор $M_{0}$ имеет либо нетривиальную $\pi _{2}$, либо нетривиальную $\pi _{3}$.
\end{enumerate}
}
\end{corollary}

\begin{myproof} 
Повторное применение Теоремы 0.3 показывает, что (1) влечёт (2). 
Импликация (2) влечёт (3), что непосредственно следует из теоремы ван Кампена. То, что 
(3) влечёт (1), доказывается Теоремой 0.4. Таким образом, (1), (2) и (3) эквивалентны.
Поскольку трёхмерные сферические пространственные формы и расслоения $S^{2}$ над $S^{1}$ 
очевидно являются простыми, (2) влечёт (4). Остаётся показать, что (4) влечёт (3).
Рассмотрим многообразие $M$, удовлетворяющее (4), простой фактор $P$ многообразия $M$, и его 
универсальное накрытие $\tilde{P}$.
Сначала предположим, что $\pi_{2} (P)=\pi_{2} (\tilde{P})$ тривиальна. Тогда, по гипотезе,
$\pi_{3} (P)=\pi_{3} (\tilde{P})$ нетривиальна. По теореме Хуревица это означает, что 
$H_{3} (\tilde{P})$ нетривиальна, а значит, $\tilde{P}$ является компактным, просто связным 
трёхмерным многообразием. Следовательно, $\pi_{1} (P)$ конечна.
Теперь предположим, что $\pi_{2} (P)$ нетривиальна. Тогда $P$ не диффеоморфно $\mathbb{RP}^3$. 
Поскольку $P$ является простым и не содержит локально разделяющихся $\mathbb{RP}^2$, оно не содержит 
вложенных $\mathbb{RP}^2$. По теореме о сферах в $P$ существует вложенная двумерная сфера, которая 
гомотопически нетривиальна. Поскольку $P$ просто, эта сфера не может разделять, поэтому 
рассечение $P$ вдоль этой сферы приводит к связному многообразию $P_{0}$ с двумя краевыми сферами. 
Поскольку $P_{0}$ является простым, оно диффеоморфно $S^{2} \times I$, а следовательно, 
$P$ диффеоморфно расслоению двумерной сферы над окружностью.
\end{myproof}

\begin{remark} (i) Использование теоремы о сферах не является необходимым в приведённом 
выше доказательстве, поскольку мы фактически доказываем, что если каждый простой фактор 
$M$ имеет нетривиальную $\pi_{2}$ или $\pi_{3}$, то поток Риччи с хирургией, начальным многообразием 
которого является $(M,g_{0})$, прекращает своё существование за конечное время. Фактически, 
теорема о сферах для замкнутых трёхмерных многообразий вытекает из изложенных здесь результатов.\\
(ii) Если начальное многообразие проще, то и все срезы по времени проще: если $(M,G)$ — это 
поток Риччи с хирургией, начальное многообразие которого является простым, то каждый срез 
по времени представляет собой несвязное объединение компонент, причём все, кроме, возможно, 
одной, диффеоморфны трёхмерной сфере, а если есть одна, которая не диффеоморфна трёхмерной 
сфере, то она диффеоморфна начальному многообразию. Если начальное многообразие $M_{0}$   
является просто связным, то каждая компонента каждого среза по времени $M_{T}$ должна быть 
просто связной, и, следовательно, \textit{апостериори} каждый срез по времени представляет собой 
несвязное объединение многообразий, диффеоморфных трёхмерной сфере. Аналогично, если 
начальное многообразие имеет конечную фундаментальную группу, то каждая связная компонента 
каждого среза по времени либо просто связна, либо имеет ту же фундаментальную группу, что 
и начальное многообразие.\\
(iii) Заключение этого результата является естественным обобщением вывода Гамильтона при 
анализе потока Риччи на многообразиях с положительной кривизной Риччи в \cite{c29}. А именно, 
при выполнении соответствующих условий в процессе эволюции потока Риччи с хирургией 
многообразие распадается на компоненты, каждая из которых исчезает за конечное время. 
Когда компонент исчезает в некоторый конечный момент времени, метрика на этом компоненте 
достаточно хорошо контролируется, чтобы показать, что исчезающий компонент допускает 
неплоскую, однородную риманову метрику с неотрицательной секционной кривизной, то есть 
метрику, локально изометричную либо круглой $S^{3}$, либо произведению круглой $S^{2}$ с обычной 
метрикой на $\mathbb{R}$. Наличие такой метрики на компоненте сразу даёт топологический вывод из 
Теоремы 0.1 для этого компонента, а именно, что он диффеоморфен трёхмерной сферической 
форме пространства, $S^{2} \times S^{1}$, неориентируемому расслоению двумерной сферы над $S^{1}$ или 
$\mathbb{RP}^{3}\# \mathbb{RP}^{3}$. Главным отличием между этими двумя результатами является то, что условие 
Гамильтона является геометрическим (положительная кривизна Риччи), тогда как условие 
Перельмана относится к теории гомотопий (информация о фундаментальной группе).\\
(iv) Также стоит отметить, что из Следствия 0.5 вытекает: многообразия, удовлетворяющие 
четырём эквивалентным условиям из этого следствия, которые именно что замкнутые, связные и 
трёхмерные, допускающие риманову метрику с положительной скалярной 
кривизной (см. \cite{c62} и \cite{c26}).
\end{remark}

Можно использовать поток Риччи для более общего исследования трёхмерных многообразий, 
чем то, которое мы проводим здесь. Существует гипотеза, выдвинутая Тёрстоном, см. \cite{c69}, 
известная как Гипотеза Геометризации Тёрстона или просто Гипотеза Геометризации для 
трёхмерных многообразий. Она предполагает, что каждое 3-многообразие, не содержащее 
локально разделяющих $\mathbb{RP}^2$ (в частности, каждое ориентируемое 3-многообразие), является 
связаной суммой простых 3-многообразий, каждое из которых допускает разложение 
вдоль несжимаемых\footnote{То есть, вложенные с помощью отображения, инъективного на $\pi_{1}$} 
(не стягиваемых?) торов на части, которые допускают локально однородные геометрии 
конечного объёма. За исключением вопросов о конечных объёмах решёток в $SL_{2}(\mathbb{C})$, 
доказательство этой гипотезы приводит к полному классификационному результату для 
3-многообразий без локально разделяющих $\mathbb{RP}^2$, а также к полной классификации всех 
ориентируемых 3-многообразий. (См. обзорную статью Питера Скотта \cite{c63}.) 
Путём перехода к двойному покрытию по ориентации и работы с эквивариантными 
методами эти результаты могут быть распространены на все 3-многообразия.

Перельман в \cite{c55} представил результаты, которые подразумевают положительное разрешение 
Гипотезы Геометризации Тёрстона. Предложенное Перельманом доказательство Гипотезы 
Геометризации Тёрстона в существенной степени зависит от Теоремы 0.3, а именно от 
существования потока Риччи с хирургией для всего положительного времени. Однако оно 
также включает дальнейший анализ пределов этих потоков Риччи, когда время стремится 
к бесконечности. Этот дальнейший анализ включает аналитические аргументы, изложенные 
в Разделах 6 и 7 второго документа Перельмана (\cite{c55}), следуя более ранней работе 
Гамильтона (\cite{c36}) в более простом случае ограниченной кривизны. Они также включают 
результат (Теорема 7.4 из \cite{c55}) из теории многообразий с локально ограниченной снизу 
кривизной, которые коллапсируют, связанный с результатами Шиоя-Ямагучи \cite{c67}. Эти 
результаты Шиоя-Ямагучи в свою очередь опираются на более раннюю неопубликованную 
работу Перельмана, доказывающую так называемую «Теорему о стабильности». Недавно 
Капович, \cite{c43}, опубликовал препринт, в котором представлено доказательство результата 
о стабильности. Мы исследуем другой подход, предложенный Перельманом в \cite{c55}, который 
избегает использования теоремы о стабильности, см. \cite{c44} и \cite{c51}. Мы считаем, что 
результаты коллапсирования, необходимые для гипотезы геометризации, уже в наличии, 
но прежде чем можно будет сделать окончательное заявление о том, что гипотеза 
геометризации решена, эти аргументы должны быть подвергнуты тому же тщательному 
анализу, которому подверглись доказательства гипотезы Пуанкаре. Этот процесс в 
настоящее время продолжается.

В этой книге мы не пытаемся разъяснить какие-либо результаты, помимо Теоремы 0.3, 
описанных в предыдущем абзаце, которые необходимы для Гипотезы Геометризации. 
Вместо этого мы ограничиваемся представлением доказательства Теоремы 0.1 выше, 
которое, как мы указали, касается начальных римановых многообразий, для которых 
поток Риччи с хирургией исчезает после конечного времени. В настоящее время мы 
готовим подробное доказательство, предложенное Перельманом, для дальнейших 
результатов, которые завершат доказательство Гипотезы Геометризации.

Как должно быть ясно из вышеизложенного обзора, аргумент Перельмана не возник в 
вакууме. Во-первых, он находится в контексте общей теории римановых многообразий. 
В частности, различные понятия сходимости последовательностей многообразий играют 
ключевую роль. Наиболее важной является геометрическая сходимость (гладкая 
сходимость на компактных подмножествах). Еще более важно, что аргумент Перельмана 
находится в контексте теории уравнения потока Риччи, введенного Ричардом 
Гамильтоном и широко изученного им и другими учеными. Перельман использует почти 
все ранее установленные результаты для трехмерных потоков Риччи. Единственным 
исключением являются предложенные Гамильтоном результаты классификации для 
трехмерных сингулярностей. Они были заменены сильным качественным описанием 
развития сингулярностей для потоков Риччи на компактных трехмерных многообразиях, 
предложенным Перельманом.

Первые пять глав книги обозривают необходимый фоновый материал из этих двух тем. 
Главы с 6 по 11 объясняют достижения Перельмана. В главе 12 мы вводим стандартное 
решение, которое представляет собой многообразие, сконструированное вручную, 
которое «вставляется» при выполнении хирургии. Главы с 13 по 17 подробно описывают 
процесс хирургии и доказывают основные аналитические и топологические оценки, 
которые необходимы для того, чтобы показать, что процесс можно продолжать для 
всего положительного времени. В конце главы 17 мы установили теорему 0.3. Глава 18 
обсуждает результат о исчезновении за конечное время. Глава 19 — это приложение с 
некоторыми топологическими результатами, которые были необходимы для анализа 
хирургии в главах 13-17.
\subsection{Основы геометрии Римана}

\textbf{2.1. Объем и радиус инъективности.} Одним важным общим понятием, которое 
используется в дальнейшем, является понятие того, что многообразие не схлопывается 
в какой-то точке. Пусть у нас есть точка $x$ в полном римановом $n$-мерном многообразии. 
Мы говорим, что многообразие \textit{$\kappa $-не-коллапсируемо} в точке $x$, если 
выполняется следующее условие: для любого $r$, при котором норма тензора кривизны Римана
$| Rm| $ не превышает $r^{-2}$ во всех точках метрического шара $B(x,r)$ радиуса $r$, центрированного 
в $x$, выполняется Vol$B(x,r)\geq \kappa r^{n}$. Существует связь между этим понятием и радиусом 
инъективности многообразия $M$ в точке $x$. А именно, если $| Rm| \leq r^{-2}$ на $B(x,r)$ и если 
$B(x,r)$ \textit{$\kappa $-не-коллапсируемо}, то радиус инъективности многообразия $M$ в точке 
$x$ больше или равен положительной константе, которая зависит только от $r$ и $\kappa $. 
Преимущество работы с условием объема, не приводящим к схлопыванию, заключается в том, 
что, в отличие от радиуса инъективности, существует простое уравнение для эволюции объема 
под действием потока Риччи.

Еще одним важным общим результатом является теорема о сравнении объемов Бишопа-Громова, 
которая утверждает, что если кривизна Риччи полного риманова $n$-мерного многообразия $M$ 
ограничена снизу постоянной $(n-1)K$, то для любой точки $x\in M$ отношение объема шара 
$B(x,r)$ к объему шара радиуса $r$ в пространстве постоянной кривизны $K$ является 
невозрастающей функцией, предел которой при $r\rightarrow 0$ равен $1$. 

Все эти базовые факты из римановой геометрии рассматриваются в первой главе.\\


\textbf{2.2. Многообразия с неотрицательной кривизной.} По причинам, которые должны стать 
ясными из вышеизложенного и, в любом случае, станут еще более очевидными вскоре, 
многообразия с неотрицательной кривизной играют чрезвычайно важную роль в анализе 
потоков Риччи с хирургией. Нам нужно несколько общих результатов о них. 
Первый — это теорема о душе для многообразий с неотрицательной секционной кривизной. 
\textit{Душа} — это компактное, тотально геодезическое подмногообразие. Все многообразие 
диффеоморфно полному пространству векторного расслоения над любой из своих душ. 
Если некомпактное $n$-мерное многообразие имеет положительную секционную кривизну, 
то любая душа для него — это точка, и, в частности, это многообразие диффеоморфно 
евклидову пространству. Кроме того, функция расстояния $f$ от души имеет свойство, 
что для каждого $t > 0$ прообраз $f^{-1} (t)$ гомеоморфен ($n-1$)-сфере, а прообраз под 
этой функцией расстояния любого недегенерированного интервала $I\subset R^{+}$ гомеоморфен 
$S^{n-1} \times I$.
\subsection{Основы потока Риччи}

Гамильтон \cite{c29} ввёл уравнение потока Риччи.
\[
\frac{\partial g(t)}{\partial t} = -2Ric(g(t))
\]
Это уравнение эволюции для однопараметрического семейства 
римановых метрик $g(t)$ на гладком многообразии $M$. 
Уравнение потока Риччи является слабо параболическим и 
строго параболическим с учётом «группы калибровок», 
которая представляет собой группу диффеоморфизмов 
исходного гладкого многообразия. 
Это уравнение следует рассматривать как нелинейную, 
тензорную версию уравнения теплопроводности. 
Из него можно вывести уравнение эволюции для римановой метрики, 
тензора Риччи и функции скалярной кривизны. 
Все эти уравнения являются параболическими. 
Например, уравнение эволюции для скалярной кривизны 
$R(x,t)$ выглядит следующим образом:
\begin{equation}
    \frac{\partial R}{\partial t}(x,t) = \bigtriangleup R(x,t)+2{| Ric(x,t)|}^2
    \label{eqn0.1}
\end{equation}
иллюстрируя сходство с уравнением теплопроводности. 
(Здесь $\bigtriangleup$ — это лапласиан с неположительным спектром.)\\

\textbf{3.1. Первые результаты.} Конечно, первые результаты, которые нам нужны, 
— это уникальность и существование решений уравнения потока Риччи для 
компактных многообразий в короткий промежуток времени. Эти результаты 
были доказаны Гамильтоном (\cite{c29}) с использованием теоремы обратной функции 
Нэша-Мозера (\cite{c28}). Эти результаты являются стандартными для строго 
параболических уравнений. На данный момент существует достаточно стандартный 
метод работы «модуло» группы калибровок (группы диффеоморфизмов), что 
позволяет достичь строго параболической ситуации, в которой применяются 
классические результаты существования, уникальности и гладкости. Метод для 
уравнения потока Риччи называется «трюк де Тёрка».

Существует также результат, который позволяет сшить локальные решения 
$(U,g(t)),a\leq t\leq b$ и $(U,h(t)),b\leq t\leq c$ в одно гладкое решение, 
определённое на интервале $a\leq t\leq c$, при условии, что 
$g(b)=h(b)$.

Заданное уравнение потока Риччи $(M,g(t))$ можно всегда сдвинуть, 
заменив $t$ на $t+t_0$ для некоторого фиксированного $t_0$, что даст 
новый поток Риччи. Также можно масштабировать его на любое 
положительное число $Q$, задав $h(t)=Q g(Q^{-1} t)$, 
что создаст новый поток Риччи.\\

\textbf{3.2. Градиентные сжимающиеся солитоны.} Предположим, что 
$(M,g)$ — полное риманово многообразие, и существует константа 
$\lambda >0$, такая что:
\[
    Ric(g)=\lambda g
\]
В этом случае легко увидеть, что существует поток Риччи, задаваемый:
\[
    g(t)=(1-2\lambda t)g
\]
В частности, все метрики в этом потоке отличаются на постоянный множитель, 
зависящий от времени, и метрика является убывающей функцией времени. 
Такие решения называются \textit{сжимающимися солитонами}. Примеры включают 
компактные многообразия с постоянной положительной кривизной Риччи.

Существует тесно связанный с ними, но более общий класс примеров: 
\textit{градиентные сжимающиеся солитоны}. Предположим, что 
$(M,g)$ — полное риманово многообразие, и существует константа 
$\lambda >0$ и функция $f:M \rightarrow \mathbb{R}$, удовлетворяющие следующему условию:
\[
    Ric(g)=\lambda g - {Hess}^g f
\]
В данном случае существует поток Риччи, представляющий собой 
семейство сжимающихся метрик после применения обратного 
преобразования посредством одномерного семейства диффеоморфизмов, 
порожденного зависящим от времени векторным полем 
$\frac{1}{1-2\lambda t}{\bigtriangledown}_g f$. 
Примером градиентного убывающего солитона является многообразие 
$S^2 \times \mathbb{R}$ с семейством метрик, представляющим 
собой произведение семейства сжимающихся сферических метрик на 
$S^2$ и семейства постоянных стандартных метрик на $\mathbb{R}$. 
Функция $f$ в этом случае равна $s^2/4$, 
где $s$ — евклидова координата на $\mathbb{R}$.\\

\textbf{3.3. Управление высшими производными кривизны.} Теперь 
давайте обсудим результаты гладкости для геометрических пределов. 
Общий результат в этом направлении — теорема Ши, см.\cite{c65, c66}. 
Это стандартный тип результата для параболических уравнений. 
Конечно, ситуация здесь осложняется существованием группы калибровок. 
В общем, теорема Ши говорит следующее. Пусть  
$B(x,t_0,r)$ — метрический шар в $(M,g(t_0))$, с центром в точке  
$x$ и радиусом $r$. Если мы можем контролировать норму тензора 
кривизны в обратной окрестности формы $B(x,t_0,r)\times [0,t_0]$, 
тогда для любого $k>0$ мы можем контролировать $k$-ую 
ковариантную производную кривизны в 
$B(x,t_0,r/2^k)\times [0,t_0]$ как константу превышающую 
$t^{k/2}$. Этот результат имеет много важных последствий для 
нашего исследования, так как он говорит нам, что геометрические 
пределы — это пределы обладающие гладкостью. 
Возможно, первое важное утверждение, которое 
стоит выделить, — это факт (установленный ранее Хэмилтоном), 
что если $(M,g(t))$ — это поток Риччи, 
определенный на $0 \leq t< T< \infty$, и если тензор кривизны Риманна 
ограничен на протяжении всего потока, тогда поток Риччи 
продолжается после времени $T$. 

В третьей главе этот материал пересматривается, и при 
необходимости представлены небольшие варианты результатов и 
аргументов из литературы.\\

\textbf{3.4. Обобщение потоков Риччи.} Поскольку мы не 
можем ограничиться только потоками Риччи, и должны 
рассматривать более общие объекты, такие как потоки Риччи 
с хирургией, важно установить основные аналитические 
результаты и оценки в контексте, более общем, чем потоки 
Риччи. Мы решили сделать это в контексте обобщенных 
потоков Риччи.

Обобщенный трехмерный поток Риччи состоит из гладкого 
четырехмерного многообразия $\mathcal{M}$ (пространства-времени) 
с функцией времени $\textbf{t}:\mathcal{M} \rightarrow \mathbb{R}$ 
и гладким векторным полем $\chi$. 
Такими, что должны выполнятся следующие условия:
\begin{enumerate}
    \item Каждое $x\in \mathcal{M}$ имеет окрестность вида $U \times J$, где 
    $U$ — открытое подмножество в $\mathbb{R}^3$, а $J\subset \mathbb{R}$ — интервал, 
    в котором $\textbf{t}$ является проекцией на $J$, а $\chi$ — единичное 
    векторное поле, касательное к одномерному слоению 
    $\{ u\} \times J$, направленному в сторону возрастания $\textbf{t}$.
    Мы называем $\textbf{t}^{-1}(t)$ $t$-временным срезом. Это гладкое 3-многообразие.
    \item Образ $\textbf{t}(\mathcal{M})$ — это связный интервал $I$ в $\mathbb{R}$, возможно бесконечный. 
    Граница $\mathcal{M}$ — это прообраз границы $I$ при обратном отображении $\textbf{t}$.
    \item Множество уровней  $\textbf{t}^{-1}(t)$  образуют слоение коразмерности один 
    на $\mathcal{M}$, называемое горизонтальным слоением, где границы компонентов 
    $\mathcal{M}$ являются листьями.
    \item Существует метрика $G$ на горизонтальном распределении, 
    т.е. на распределении касательном к множеству уровней $\textbf{t}$.
    Эта метрика задаёт риманову метрику на каждом $t$ временном срезе, 
    которая изменяется гладко по мере изменения самого временного среза.
    Мы определяем кривизну $G$ в точке $x \in \mathcal{M}$ как кривизну 
    римановой метрики, индуцированной $G$ на временном срезе $\mathcal{M}_t$ в $x$.
    \item Благодаря первому свойству интегральные кривые векторного 
    поля $ \chi $ сохраняют горизонтальное слоение, а значит, и горизонтальное 
    распределение. Таким образом, мы можем взять производную Ли от $ G $ 
    вдоль $ \chi $. Уравнение потока Риччи тогда принимает вид:
    \[
    \mathcal{L}_{\chi}(G)=-2Ric(G)
    \]
\end{enumerate}

Локально в пространстве-времени горизонтальная метрика 
представляет собой простое гладко меняющееся семейство 
римановых метрик на многообразии фиксированной гладкости, 
а уравнение эволюции является обычным уравнением потока 
Риччи. Это означает, что стандартные формулы для эволюции 
кривизн, а также большая часть аналитического анализа 
потоков Риччи сохраняются в этом обобщённом контексте. 
В конечном счёте поток Риччи с хирургией является более 
сингулярным типом пространства-времени, но он будет иметь 
открытое плотное подмножество, которое является 
обобщённым потоком Риччи, и все аналитические оценки 
выполняются в этом открытом подмножестве.

Понятие канонических окрестностей имеет смысл в контексте 
обобщённых потоков Риччи. Существует также понятие сильной 
$ \epsilon $-шейки. Рассмотрим вложение 
$ \psi : S^2 \times (-\epsilon^{-1}, \epsilon^{-1}) \times (-1, 0] $ 
в пространство-время такое, что функция времени испытывает pullback 
на проекцию в $ (-1, 0] $, а векторное поле $ \chi $ 
pullback на $ \partial / \partial t $. Если существует 
такое вложение в соответствующим образом смещённую и 
перемасштабированную версию исходного обобщённого потока Риччи, 
что pullback перемасштабированной горизонтальная метрики 
находится в пределах $ \epsilon $ в $ C^{[1/\epsilon]} $-топологии 
произведения сжимающегося семейства окружностей $ S^2 $ с евклидовой 
метрикой на $ (-\epsilon^{-1}, \epsilon^{-1}) $, то мы говорим, 
что $ \psi $ является сильной $ \epsilon $-шейкой в обобщённом 
потоке Риччи.

\subsection{Достижения Перельмана}
текст
\subsection{Стандартное решение и процесс хирургии}
текст
\subsection{Расширение потоков Риччи с хирургией}
текст
\subsection{Исчезновение за конечное время}
текст
\subsection{Благодарности}
текст
\subsection{Список связанных статей}
текст


\newpage
\section{Введение}
\subsection{Метрика Римана и связность Леви-Чивиты}
текст
\subsection{Подглава}
текст
\subsection{Подглава}
текст
\subsection{Подглава}
текст
\subsection{Подглава}
текст
\subsection{Подглава}
текст
\subsection{Исчезновение за конечное время}
текст
\subsection{Благодарности}
текст
\subsection{Список связанных статей}
текст


\newpage
\section{Многообразия с неотрицательной кривизной}
\subsection{Подглава}
текст
\subsection{Подглава}
текст
\subsection{Подглава}
текст
\subsection{Подглава}
текст
\subsection{Подглава}
текст
\subsection{Подглава}
текст
\subsection{Коэффициенты прямой разницы}
текст
\newpage
\section{Основы потока Риччи}
\subsection{Подглава}
текст
\subsection{Подглава}
текст
\subsection{Подглава}
текст
\subsection{Подглава}
текст
\subsection{Подглава}
текст
\subsection{Изменение расстояния под действием потока Риччи}
текст
\subsection{Оценки производных Ши}
текст
\subsection{Обобщенные потоки Риччи}
текст
\newpage
\section{Принцип максимума}
\subsection{Подглава}
текст
\subsection{Подглава}
текст
\subsection{Подглава}
текст
\subsection{Подглава}
текст
\subsection{Подглава}
текст

\newpage
\section{Название раздела нового}
\subsection{Подглава}
текст
\subsection{Подглава}
текст
\subsection{Подглава}
текст
\subsection{Подглава}
текст
\subsection{Подглава}
текст
\subsection{Подглава}
текст
\newpage
\section{Геометрический подход к потоку Риччи через сравнения}
\subsection{Подглава}
текст
\subsection{Подглава}
текст
\subsection{Подглава}
текст
\subsection{Подглава}
текст
\subsection{Подглава}
текст
\subsection{Подглава}
текст
\subsection{Сокращённый объём}
текст
\newpage
\section{Полные потоки Риччи с ограниченной кривизной}
\subsection{\texorpdfstring{Функции $L_{x}$ и $l_{x}$}{Функции Lx и lx}}
Текст о функциях $L_{x}$ и $l_{x}$
\subsection{Подглава}
текст
\subsection{Подглава}
текст
\newpage
\section{Результаты о несхлопывающихся многообразиях}
\subsection{Подглава}
текст
\subsection{Подглава}
текст
\newpage
\section{Название раздела нового}
\subsection{Подглава}
текст
\subsection{Подглава}
текст
\subsection{Подглава}
текст
\subsection{Подглава}
текст
\subsection{Подглава}
текст
\subsection{Подглава}
текст
\newpage
\section{Название раздела нового}
\subsection{Подглава}
текст
\subsection{Подглава}
текст
\subsection{Подглава}
текст
\subsection{Подглава}
текст
\subsection{Подглава}
текст
\subsection{Подглава}
текст
\newpage
\section{Основы потока Риччи}
\subsection{Подглава}
текст
\subsection{Подглава}
текст
\subsection{Подглава}
текст
\subsection{Подглава}
текст
\newpage
\section{Стандартное решение}
\subsection{Подглава}
текст
\subsection{Подглава}
текст
\subsection{Подглава}
текст
\subsection{Подглава}
текст
\subsection{Подглава}
текст
\subsection{Подглава}
текст
\subsection{Некоторые следствия}
текст
\newpage
\section{\texorpdfstring{Хирургия на $\delta$-шейке}{Хирургия на δ-шейке}}
\subsection{Подглава}
текст
\subsection{Предварительные вычисления}
текст
\subsection{Подглава}
текст
\subsection{Подглава}
текст
\newpage
\section{Название раздела нового}
\subsection{Подглава}
текст
\subsection{Обобщённое уравнение потока Риччи}
текст
\subsection{Подглава}
текст
\subsection{Подглава}
текст
\subsection{Подглава}
текст
\subsection{Подглава}
текст
\newpage
\section{Контролируемые потоки Риччи с хирургией}
\subsection{Подглава}
текст
\subsection{Подглава}
текст
\subsection{Подглава}
текст
\subsection{Подглава}
текст
\subsection{Утверждения о существовании потока Риччи с хирургией}
текст
\subsection{Подглава}
текст
\newpage
\section{Доказательство несхлопывания}
\subsection{Формулировка результата о несхлопывании}
текст
\subsection{Подглава}
текст
\subsection{Подглава}
текст
\subsection{Подглава}
текст
\subsection{Подглава}
текст
\subsection{Подглава}
текст
\newpage
\section{Название раздела нового}
\subsection{Подглава}
текст
\subsection{Время хирургии не накапливается}
текст
\subsection{Подглава}
текст
\subsection{Подглава}
текст
\subsection{Подглава}
текст
\subsection{Подглава}
текст
\newpage
\section{Название раздела нового}
\subsection{Подглава}
текст
\subsection{Подглава}
текст
\subsection{Подглава}
текст
\subsection{Подглава}
текст
\subsection{Подглава}
текст
\subsection{Подглава}
текст
\newpage
\section{Название раздела нового}
\subsection{Подглава}
текст
\subsection{Подглава}
текст
\subsection{Подглава}
текст
\subsection{\texorpdfstring{Области, покрытые $\epsilon$-шейками и $(C,\epsilon)$-капами}{Области, покрытые ϵ-шейками и (C,ϵ)-капами}}
Области, покрытые $\epsilon$-шейками и $(C,\epsilon)$-капами
\subsection{Подглава}
текст
\subsection{Подглава}
текст
\newpage
\section*{Список литературы}
\addcontentsline{toc}{section}{Список литературы}
Основной текст списка литературы

\end{document}